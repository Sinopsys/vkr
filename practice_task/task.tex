\documentclass[russian, a4paper, 12pt]{article}
\usepackage[utf8]{inputenc}
\usepackage[russian]{babel}
\usepackage{setspace}
\newcommand\tab[1][1cm]{\hspace*{#1}}
\usepackage[left=20mm, top=15mm, right=15mm, bottom=15mm, nohead, footskip=10mm]{geometry} % настройки полей документа
\usepackage{tocloft}
\usepackage{hyperref}
\hypersetup{
    colorlinks,
    citecolor=black,
    filecolor=black,
    linkcolor=black,
    urlcolor=black
}

%%% Работа с русским языком
\usepackage{cmap}                   % поиск в PDF
\usepackage[T2A]{fontenc}           % кодировка
\usepackage[utf8]{inputenc}         % кодировка исходного текста

%%% Повтор команды несколько раз через     \Repeat{n-times}{\command}
\usepackage{expl3}
\ExplSyntaxOn
\cs_new_eq:NN \Repeat \prg_replicate:nn
\ExplSyntaxOff
\usepackage{tocloft}
\renewcommand{\cftsecleader}{\cftdotfill{\cftdotsep}}

\begin{document}
\thispagestyle{empty}

\begin{center}
    \textbf{
        Правительство Российской Федерации\\
        Федеральное государственное автономное образовательное учреждение\\
        высшего образования\\
        «Национальный исследовательский университет\\
    «Высшая школа экономики» }

    \begin{center}  Факультет компьютерных наук\\
        Департамент  программной инженерии\\
        Образовательная программа 09.03.04 «Программная инженерия»
    \end{center}

    {\hfill \break}

    {\Large\bfseries Задание на преддипломную практику}\\
\end{center}
\begin{flushleft}
    {\bfseries Студент группы БПИ151: } \underline{Куприянов Кирилл Игоревич}\\
    {\hfill \break}
    {\bfseries Место прохождения практики: } Департамент Программной Инженерии, Факультет Компьютерных Наук, НИУ ВШЭ\\
    {\bfseries Сроки прохождения: } 01.04.2019 --- 28.04.2019\\
    {\bfseries Задание: } Криптографические алгоритмы и протоколы для распределенных реестров\\
    {\bfseries Задачи:}
    \begin{itemize}
        \item Выявить популярные распределённые реестры и выделить криптографические алгоритмы в них
        \item Изучить выявленные алгоритмы и способы их классификации
        \item Замерить параметры алгоритмов
        \item Классифицировать их согласно выявленным метрикам
        \item Создать обновлённую на 2019 год классификацию алгоритмов и
              протоколов в распределённых реестрах
        \item Реализовать Python3.6.5 библиотеку, содержащую все изученные
              алгоритмы и протоколы, позволяющую создавать распределённый реестр
              в учебных целях
        \item Опубликовать библиотеку в PyPi
    \end{itemize}
\end{flushleft}
\Repeat{2}{\hfill \break}
{\bfseries Критерий успешного прохождения: } Положительный отзыв руководителя практики от организации прохождения\\
{\bfseries Форма отчётности: } Отчёт по практике\\

{\hfill \break}
Руководитель практики от организации
\hspace{1cm}\underline{\hspace{3cm}}\hspace{1cm} С.М. Авдошин

{\hfill \break}
Задание принял на исполнение
\hspace{2.7cm}\underline{\hspace{3cm}}\hspace{1cm} К.И. Куприянов

{\hfill \break}
Руководитель практики, \\к.т.н., профессор Департамента\\программной инженерии ФКН
\hspace{2.8cm}\underline{\hspace{3cm}}\hspace{1cm} С.М. Авдошин
% Должность \underline{\hspace{15cm}}\\
% \newline
% ФИО \underline{\hspace{8cm}}\hspace{0.3cm}
% Оценка \underline{\hspace{6.2cm}}\\
% \newline
% Дата \underline{\hspace{4cm}}\hspace{4.35cm}
% Подпись \underline{\hspace{6cm}}
\Repeat{1}{\hfill \break}

\begin{center}
    Москва, 2019г.
\end{center}

\end{document}


% EOF
