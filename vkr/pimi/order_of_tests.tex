%=========================================
\subsection{Параметры технических средств, используемых во время испытаний}
Для испытания программы необходимо учесть следующие системные требования:
\begin{my_enumerate}
    \item Персональный компьютер со следующими минимальными требованиями:
        \begin{my_enumerate}
            \item Операционная GNU/Linux версии ядра 4.15.0-47-generic и выше
            \item 64-разрядный (x64) процессор
            \item 1ГБ оперативной памяти (ОЗУ)
            \item 100 МБ свободного места на внутреннем накопителе
        \end{my_enumerate}
    \item Интерпретатор Python3.6.5 и выше
\end{my_enumerate}


%=========================================
\subsection{Порядок проведения испытаний}
Испытания проводятся поэтапно, друг за другом, в следующем порядке:
\begin{my_enumerate}
    \item Испытание выполнения требований к программной документации
    \item Испытание выполнения требований к функциональным характеристикам программы, надежности и корректности ее работы
    \item Испытание выполнения требований к временным характеристикам
\end{my_enumerate}


%=========================================
\subsection{Условия проведения испытаний}

\subsubsection{Требования к численности и квалификации персонала}
Минимальное количество персонала, требуемого для работы программы: 1 оператор.
Пользователь данного программного продукта должен разбираться в командной
строке (shell) GNU/Linux, иметь базовые навыки в командах, уметь устанавливать
и удалять программы, запускать их. Перед использованием программы
пользователь должен быть заранее проинструктирован и уведомлен о составе
выполняемых функций и других характеристиках приложения, а так же сопровождён
необходимой технической документацией.

