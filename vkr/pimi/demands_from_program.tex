\subsection{Требования к функциональным характеристикам}
\subsubsection{Требования к составу выполняемых функций. Клиентская часть (Android приложение).}
\begin{my_enumerate}
\item Возможность просмотра списка доступных магазинов с акционными товарами
\item Представление текущих акций для конкретного магазина:
    \begin{my_enumerate}
    \item В виде общего списка
    \item По категориям
    \end{my_enumerate}
\item Постепенная загрузка товаров магазинов (по страницам) для экономии трафика и меньшей нагрузкой на мобильное устройство
\item Регистрация через мобильное приложение
\item Вход в аккаунт через мобильное приложение
\item Возможность смены аккаунта
\item Возможность создания списков покупок c разными названиями
\item Возможность удаления списка покупок
\item Возможность добавления товара в список покупок
\item Возможность удаления товара из списка покупок
\item Возможность добавления в список покупок пользовательских товаров, котороых нет в магазине (см. терминологию)
\item Возможность просмотра подобранных программой товаров согласно запросу пользователя
\item Добавление подобранных товаров в список покупок
\item Предварительное отображение элементов каждого списка покупок до их открытия
\item Отображение всплывающих подсказок при долгом нажатии на элементы управления button (кнопка)
\item При отсутствии интернет-соединения перенаправление в настройки сети для последующего включения интернета
\item Отображение индикатора процесса загрузки данных с сервера
\item Реализация обучающего фрагмента в разделе help (см. терминологию), содержащего руководство пользователя по управлению программой
\end{my_enumerate}

\subsubsection{Требования к составу выполняемых функций. Серверная часть.}
\begin{my_enumerate}
\item Crawling веб-страниц для сбора актуальной информации об акционных товарах
\item Добавление товаров в базу данных посредством отправки запросов REST API (REST API и база данных реализованы напарником)
\item Запись акционных товаров во всех магазинах в JSON файл
\item Email уведомления администраторам сервиса об ошибках и неполадках в работе сервера
\item Ежедневное обновление акций для поддержания актуальности
% \item Реализация панели администратора для управления web-crawler'ом
\end{my_enumerate}

\subsection{Требования к временным характеристикам}
При скорости интернет соединения 30Мбит/с:
\begin{my_enumerate}
\item Загрузка одной страницы с товарами -- не более 6 секунд
\item Отправка списка покупок на сервер -- не более 3 секунд
\end{my_enumerate}

\subsection{Требования к интерфейсу}
\begin{my_enumerate}
\item Совместимость с графической подсистемой ОС Android {\textregistered}
\item Язык интерфейса: русский и английский, в зависимости от выбранного в настройках мобильного устройства
\item Оформление программы в стиле соответствующему guideline от Google: \url{http://material.io/guidelines/style/color.html}
\item Интуитивная ясность конечному пользователю без наличия специального или профессионального образования
\item Полоса загрузки в центре экрана для индикации состояния скачивания данных с сервера
\end{my_enumerate}

\subsection{Требования к надежности}
\subsubsection{Обеспечение устойчивого функционирования программы}

Для надежной работы программы требуется исполнение следующих требований:
\begin{my_enumerate}
\item Обеспечение поддержания заряда аккумуляторной батареи устройства на
уровне не ниже 30\%, иначе обеспечить бесперебойную подзарядку оборудования
\item Обеспечение использования лицензионного программного обеспечения
\item Обеспечение защиты операционной системы и технических средств от
вредоносного воздействия шпионских программ, компьютерных вирусов и сетевых
червей
\item Обеспечение своевременного обновления программных составляющих мобильного устройства
\item При изменении дизайна веб-сайта магазина, администратор приложения оперативно
исправляет соответствующие селекторы для кроулера
\item Раз в сутки производить бэкап всех баз данных
\end{my_enumerate}
