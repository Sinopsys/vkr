\subsection{Терминология}
\begin{description}
    \item[Распределённый реестр (Distributed Ledger)] ---
        В примитивной своей реализации это распределённая база данных между
        сетевыми узлами или вычислительными устройствами.
        Каждый из узлов может получать данные других, при этом храня полную
        копию реестра. Обновления этих узлов происходят независимо друг от
        друга.

    \item[Блокчейн] ---
        Постоянно растущий список записей, называемых блоками, которые связаны
        и защищены с помощью криптографии. Он копируется его пользователями и
        устойчив к модификации. Машина с рабочей копией называется узлом.

    \item[DAG] ---
        Направленный ациклический граф. Это ориентированный граф с данными,
        использующий топологическую сортировку (от ранних узлов к более поздним).

    \item[Биткоин (Bitcoin)] ---
        Электронная пиринговая платёжная система, используемая в качестве
        финансовой единицы (криптовалюты) одноимённую сущность. Создателем
        биткоина выступает некто Satoshi Nakomoto \cite{Nakamoto2008}.

    \item[Эфириум (Ethereum)] ---
        Открытая, общедоступная, вторая по популярности, распределенная
        вычислительная платформа на основе технологии блокчейн и операционная
        система с функциональностью смарт-контрактов
        \cite{VitalikButerin2015}

    \item[Алгоритм консенсуса] ---
        Набор математических операций, которые необходимо выполнять для
        поддержания консистентности всей сети.
\end{description}

