\subsection{Терминология}
\begin{description}
    \item[Активность (Activity)] ---
        Activity — это компонент приложения, который выдает экран, и с которым
        пользователи могут взаимодействовать для выполнения каких-либо
        действий, например набрать номер телефона, сделать фото, отправить
        письмо или просмотреть карту. 

    \item[Фрагмент (Fragment)] ---
        Фрагмент (класс Fragment) представляет поведение или часть
        пользовательского интерфейса в операции (класс Activity). В одной
        активности может быть несколько фрагментов.

    \item[crawler] ---
            Программный модуль, работающий в фоне и производящий сбор данных с сайтов
            указанных магазинов, с последующей отправкой их на сервер в формате
            JSON

    \item[Пользовательский товар] ---
            Товар, представленный в виде текста, имеющий в себе массив товаров,
            подходящих при сопоставлении названий к данному. Пример.
            Пользовательский товар ``Сок'' имеет массив сопоставившихся товаров [Сок
            Добрый 1л Яблоко; Сок J-7 апельсин с мякотью].

    \item[Spider] ---
        Часть crawler'a, отвечающая за непосредственный сбор информации с
        веб-страниц, переход между страницами и дальнейшую отправку собранных
        данных другим модулям crawler'a.

    \item[log-сообщения] ---
        Сообщения, которые выводит система для подробного отслеживания
        происходящих в ней процессах. Обычно содержит точное время процесса,
        тэг процесса, и информативное сообщение.

\end{description}

