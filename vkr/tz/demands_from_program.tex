\subsection{Требования к функциональным характеристикам}

\subsubsection{Состав выполняемых функций компоновщика}
\begin{my_enumerate}
    \item Возможность вывода на консоль вариантов выбора по категориям:
        \begin{my_enumerate}
                \item Отображение возможных структур реестра
                \item Отображение возможных типов открытости реестра
                \item Отображение возможных алгоритмов консенсуса
                \item Отображение возможных алгоритмов хэширования
                \item Отображение возможных алгоритмов генерации случайных чисел
                \item Отображение возможных алгоритмов цифровой подписи
        \end{my_enumerate}
    \item Генерирование значений для выбора ``по умолчанию''
    \item Возможность записать выбора пользователя
    % \item Возможность поиска в хранилище ссылок для конкретных алгоритмов
    % \item Возможность загрузки из общедоступных источников исходных кодов алгоритмов
    \item Возможность установки загруженных алгоритмов на ФС машины без исключительных root прав
    \item Возможность генерировать код по указанной директории
    \item Возможность замера времени работы выбранных алгоритмов
    \item Возможность просмотра справочной информации по остальным параметрам реестра
    \item Вывод информации в цвете, обозночающий степень поддержки программой алгоритма
\end{my_enumerate}

\subsubsection{Состав выполняемых функций реализации блокчейна}
\begin{my_enumerate}
    \item Возможность генерации ``адреса кошелька'' --- пары приватный + публичный ключ
    \item Использование в качестве алгоритма цифровой подписи выбранный пользователем
    \item Использование в качестве алгоритма хэширования выбранный пользователем
    \item Возможность записи данных ``адреса кошелька'' на ФС машины
    \item Возможность отправки от одного пользователя другому условное
          количество условной криптовалюты
    \item Возможность получения всей цепочки транзакций, которые были проведены
          за текущую сессию путём вызова API функции майнера
\end{my_enumerate}

\subsubsection{Требования к временным характеристикам компоновщика}
На машине с установленной ОС Ubuntu Linux (версия ядра 4.15.0-47), с
процессором Intel(R) Core(TM) i7-8550U CPU @ 1.80GHz, 16 Гб ОЗУ, текущей
утилизацией диска 0.06\%, load average за последнюю минуту 0.22, время
выполнения программы не должно превышать 1.05 секунд (без учёта на установку
пакетов алгоритмов в систему). Требования к временным характеристикам от
значений сети не зависят.

\subsubsection{Требования к временным характеристикам реализации блокчейна}
Временные показатели реализации варьируются в зависимости от выбранной
пользователем связке алгоритмов (всего 24 варианта на текущий момент),
вследствие чего, требования не предъявляются.

\subsubsection{Требования к интерфейсу компоновщика}
Данное программное средство распространяется в виде приложения командной
строки, вследствие чего требования к интерфейсу компоновщика не предъявляются.
% Настоящее программное средство (компоновщик) представлено в виде приложения
% командной строки, в котором предусмотрено взаимодействие с пользователем.
% Приложение предъявляет следующие требования к интерфейсу:
% \begin{my_enumerate}
%     \item Отображение на консоли приветственного сообщения с краткой инструкцией действий
%     \item Отображение на консоли вариантов выбора по категориям:
%         \begin{my_enumerate}
%                 \item Отображение возможных структур реестра
%                 \item Отображение возможных типов открытости реестра
%                 \item Отображение возможных алгоритмов консенсуса
%                 \item Отображение возможных алгоритмов хэширования
%                 \item Отображение возможных алгоритмов генерации случайных чисел
%                 \item Отображение возможных алгоритмов цифровой подписи
%         \end{my_enumerate}
%     \item Раскраска опций алгоритмов по степени поддержки программой
%     \item Отображение выбранных пользователем опций в цветовом форматировании
%     \item Отображение справочной информации по выбранным опциям
%     \item Логирование на уровнях INFO, DEBUG и ERROR
% \end{my_enumerate}

\subsubsection{Требования к интерфейсу реализации блокчейна}
Данное программное средство распространяется в виде приложения командной
строки, вследствие чего требования к интерфейсу реализации блокчейна не
предъявляются.


\subsection{Требования к надежности}
\subsubsection{Обеспечение устойчивого функционирования компоновщика}

Для надежной работы программы требуется исполнение следующих требований:
\begin{my_enumerate}
    \item Обеспечение поддержания заряда аккумуляторной батареи устройства
          (ноутбука) на уровне не ниже 20\%, иначе обеспечить бесперебойную
          подзарядку оборудования
    \item Обеспечение использования лицензионного программного обеспечения
    \item Обеспечение защиты операционной системы и технических средств от
          вредоносного воздействия шпионских программ, компьютерных вирусов и
          сетевых червей
    \item Обеспечение своевременного обновления программных составляющих
          устройства
\end{my_enumerate}


\subsubsection{Обеспечение устойчивого функционирования реализации блокчейна}

Для надежной работы программы требуется исполнение следующих требований:
\begin{my_enumerate}
    \item Обеспечение бесперебойного питания (UPS) оборудования; если же
          ожидается запланированное завершение выполнения программы в период
          времени, не превышающий ожидаемое время разрядки аккумулятора
          оборудования --- поддержание уровня заряда 20\%.
    \item Обеспечение использования лицензионного программного обеспечения
    \item Обеспечение защиты операционной системы и технических средств от
          вредоносного воздействия шпионских программ, компьютерных вирусов и
          сетевых червей
    \item Обеспечение своевременного обновления программных составляющих
          устройства
\end{my_enumerate}


\subsubsection{Время восстановления после отказа}
В случае возникновения сбоя, вызванного внешними факторами (непредвиденное
выключение питания, устранимые неполадки оборудования) время восстановления
программы не должно превышать суммарного затраченного времени на решение
проблем с используемым устройством и его перезагрузки. Если программа была
аварийно завершена в связи с некорректными действиями оператора, то время
восстановления программы не должно превышать времени ее повторного запуска.

\subsubsection{Отказы из-за некорректных действий оператора}
В случае установки программы на устройство, не имеющего необходимых технических
характеристик, пользователю должно сообщаться об ошибке

\subsection{Условия эксплуатации}
Пользователь данного программного продукта должен разбираться в командной
строке (shell) GNU/Linux, иметь базовые навыки в командах, уметь устанавливать
и удалять программы, запускать их. Перед использованием программы
пользователь должен быть заранее проинструктирован и уведомлен о составе
выполняемых функций и других характеристиках приложения, а так же сопровождён
необходимой технической документацией.

\subsection{Требования к составу и параметрам технических средств}
Для оптимальной работы приложения необходимо учесть следующие требования:
\begin{my_enumerate}
    \item Персональный компьютер со следующими минимальными требованиями:
        \begin{my_enumerate}
            \item Операционная GNU/Linux версии ядра 4.15.0-47-generic и выше
            \item 64-разрядный (x64) процессор
            \item 1ГБ оперативной памяти (ОЗУ)
            \item 100 МБ свободного места на внутреннем накопителе
        \end{my_enumerate}
    \item Интерпретатор Python3.6.5 и выше
\end{my_enumerate}

\subsection{Требования к информационной и программной совместимости}
Для разработки использовался язык Python 3.6.5, для написания конфигурационных
файлов --- yaml 2.1; Для установки следует воспользоваться консольной утилитой
python3-pip.

\subsection{Требования к маркировке и упаковке}
Программа поставляется в виде Python3 пакета на внешнем носителе
информации – CD/DVD диске. На нем должны содержаться программная
документация и презентация проекта.

\subsection{Требования к транспортированию и хранению}
Особые требования к транспортировке и хранению не предъявляются.

% EOF

