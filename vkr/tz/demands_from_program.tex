\subsection{Требования к функциональным характеристикам}

\subsubsection{Состав выполняемых функций}
\begin{my_enumerate}
\item Чтение конфигурационного файла по пути из аргументов командной строки
\item Валидация конфигурационного файла по типам и их значениям
\item Возможность обновления (изменения) встроенного кода алгоритмов и протоколов
\item Возможность добавления новых кодов алгоритмов и протоколов
\item Возможность выбирать алгоритмы по категориям
\item Возможность выбора алгоритмов по умолчанию
\item Возможность отменить выбор и начать процесс выбора заново без остановки
      работы программы
\item Генерация кода распределённого реестра в соответствии с выбранными
      функциями и алгоритмами
\item Возможность изменить директорию создания (инициализации) кода реестра
\item Устанавливаемость при помощи утилиты python3-pip
\end{my_enumerate}

\subsubsection{Требования к временным характеристикам}
На машине с установленной ОС Ubuntu Linux (версия ядра 4.15.0-47), с
процессором Intel(R) Core(TM) i7-8550U CPU @ 1.80GHz, 16 Гб ОЗУ  и текущей
утилизацией диска 0.16\%, время выполнения программы должно превышать 0.17
секунд. Требования к временным характеристикам от значений сети не зависят.

\subsubsection{Требования к интерфейсу}
Настоящая библиотека предъявляет следующие требования к интерфейсу:
\begin{my_enumerate}
\item Отображение на консоли всего списка алгоритмов по категориям:
    \begin{my_enumerate}
            \item Отображение возможных структур реестра
            \item Отображение возможных типов открытости реестра
            \item Отображение алгоритмов консенсуса
            \item Отображение алгоритмов хэширования
            \item Отображение алгоритмов генерации случайных чисел
    \end{my_enumerate}
\item Отображение выбранных пользователем опций в цветовом форматировании
\item Отображение справочной (help) информации по категориям алгоритмов
\item Логирование на уровнях INFO, DEBUG и ERROR
\end{my_enumerate}

\subsection{Требования к надежности}
\subsubsection{Обеспечение устойчивого функционирования программы}

Для надежной работы программы требуется исполнение следующих требований:
\begin{my_enumerate}
    \item Обеспечение поддержания заряда аккумуляторной батареи устройства
          (ноутбука) на уровне не ниже 30\%, иначе обеспечить бесперебойную
          подзарядку оборудования
    \item Обеспечение использования лицензионного программного обеспечения
    \item Обеспечение защиты операционной системы и технических средств от
          вредоносного воздействия шпионских программ, компьютерных вирусов и
          сетевых червей
    \item Обеспечение своевременного обновления программных составляющих
          устройства
\end{my_enumerate}


\subsubsection{Время восстановления после отказа}
В случае возникновения сбоя, вызванного внешними факторами (непредвиденное
выключение питания, устранимые неполадки оборудования) время восстановления
программы не должно превышать суммарного затраченного времени на решение
проблем с используемым мобильным устройством и его перезагрузки. Если программа
была аварийно завершена в связи с некорректными действиями оператора, то время
восстановления программы не должно превышать времени ее повторного запуска.

\subsubsection{Отказы из-за некорректных действий оператора}
В случае установки программы на устройство, не имеющего необходимых технических
характеристик, пользователю должно сообщаться об ошибке

\subsection{Условия эксплуатации}
Пользователь данного программного продукта должен разбираться в командной
строке (shell) GNU/Linux, иметь базовые навыки в командах, уметь устанавливать
и удалять программы, запускать их. Перед использованием программы
пользователь должен быть заранее проинструктирован и уведомлен о составе
выполняемых функций и других характеристиках приложения.

\subsection{Требования к составу и параметрам технических средств}
Для оптимальной работы приложения необходимо учесть следующие системные требо-
вания:
\begin{my_enumerate}
    \item Мобильный телефон со следующими минимальными требованиями:
        \begin{my_enumerate}
            \item Операционная GNU/Linux версии ядра 4.15.0-47-generic и выше
            \item 64-разрядный (x64) процессор
            \item 1ГБ оперативной памяти (ОЗУ)
            \item 100 МБ свободного места на внутреннем накопителе
        \end{my_enumerate}
\end{my_enumerate}

\subsection{Требования к информационной и программной совместимости}
Для разработки использовался язык Python 3.6.5, для написания конфигурационных
файлов --- yaml 2.1; Для установки следует воспользоваться консольной утилитой
python3-pip.

\subsection{Требования к маркировке и упаковке}
Программа поставляется в виде Python3 пакета на внешнем носителе
информации – CD/DVD диске. На нем должны содержаться программная
документация и презентация проекта.

\subsection{Требования к транспортированию и хранению}
Особые требования к транспортировке и хранению не предъявляются.

% EOF

