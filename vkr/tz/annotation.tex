
\tab[0.75cm] Техническое задание – это основной документ, оговаривающий набор требований и
порядок создания программного продукта, в соответствии с которым производится разработка
программы, ее тестирование и приемка.

Настоящее Техническое задание на разработку ``Клиент-Серверное Android-Приложение для Управления Скидками в Розничных Сетях'' содержит следующие разделы: ``Введение'', ``Основания для разработки'',
``Назначение разработки'', ``Требования к программе'', ``Требования к программным документам'',
``Технико-экономические показатели'', ``Стадии и этапы разработки'', ``Порядок контроля и
приемки'' и приложения.

В разделе ``Введение'' указано наименование и краткая характеристика области применения
``Клиент-Серверного Android-Приложения для Управления Скидками в Розничных Сетях''.

В разделе ``Основания для разработки'' указан документ на основании, которого ведется
разработка и наименование темы разработки.

В разделе ``Назначение разработки'' указано функциональное и эксплуатационное
назначение программного продукта.

Раздел ``Требования к программе'' содержит основные требования к функциональным
характеристикам, к надежности, к условиям эксплуатации, к составу и параметрам технических
средств, к информационной и программной совместимости, к маркировке и упаковке, к
транспортировке и хранению, а также специальные требования.

Раздел ``Требования к программным документам'' содержит предварительный состав
программной документации и специальные требования к ней.

Раздел ``Технико-экономические показатели'' содержит ориентировочную экономическую
эффективность, предполагаемую годовую потребность, экономические преимущества разработки
``Клиент-Серверного Android-Приложения для Управления Скидками в Розничных Сетях''.

Раздел ``Стадии и этапы разработки'' содержит стадии разработки, этапы и содержание
работ.

В разделе ``Порядок контроля и приемки'' указаны общие требования к приемке работы.

% Настоящий документ разработан в соответствии с требованиями:\\
% 1) ГОСТ 19.101-77 Виды программ и программных документов \cite{gost_types_of_software};\\
% 2) ГОСТ 19.102-77 Стадии разработки \cite{gost_stages_of_devel};\\
% 3) ГОСТ 19.103-77 Обозначения программ и программных документов \cite{gost_marking_software};\\
% 4) ГОСТ 19.104-78 Основные надписи \cite{gost_main_signs};\\
% 5) ГОСТ 19.105-78 Требования к программным документам \cite{gost_demands_for_docs};\\
% 6) ГОСТ 19.201-78 Техническое задание. Требования к содержанию и оформлению \cite{gost_tz}.\\

% Изменения к данному Техническому заданию оформляются согласно ГОСТ 19.603-78 \cite{gost_main_rules_change},
% Перед прочтением данного документа рекомендуется ознакомиться с терминологией,
% приведенной в Приложении 1 настоящего технического задания.
