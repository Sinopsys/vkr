\subsection{Наименование программы}
Наименование программы на русском:
``Криптографические алгоритмы и протоколы для распределенных реестров''. \\
Наименование на английском:
``Cryptographic Algorithms and Protocols for Distributed Ledgers''. \\


\subsection{Краткая характеристика}
Программа предназначена для пользователей машин на семействе ОС GNU/Linux.
Цель работы --- создать удобное приложение для автоматизации программирования,
которое генерировало бы готовый код блокчейна с использованием алгоритмов,
выбранных пользователями.

Данный продукт будет служить ``инструментарием'' для программиста или любого
другого интересующегося криптографическими алгоритмами и протоколами, который
имел бы потребность интегрировать блокчейн в своё приложение (регистрация
гостей в отеле, социальную сеть, переводы, учёт документов). Так же программа
будет полезна людям, которые хотят узнать как работают современные
распределённые реестры с рассмотренными
аспектами. Это позволит быстро получать необходимую техническую информацию,
которую с трудом можно найти в общем доступе. Программа должна предоставлять не
только генерацию кода, но и дружелюбный интерфейс командной строки, в которой
форматирование и подсветка не будут сбивать с толку неподготовленного
пользователя.\\

Главной чертой данного приложения является самоподдерживаемая система по работе
с исходными кодами алгоритмов, расположенными удалённо. А так же лёгкая,
быстрая масштабируемость и модульность программного кода.

Приложение состоит из двух компонент:
\begin{enumerate}
    \item Позволяющей сгенерировать код блокчейна с использованием выбранных
          пользователем алгоритмов
    \item Является выходом первой компоненты, и по своей сути обособленным приложением --- блокчейном
\end{enumerate}

В дальнейшем (1) будет именоваться \textbf{компоновщик}, а (2) --- \textbf{реализация блокчейна}. 
