\documentclass[encoding=utf8]{twoeskd}


\usepackage[export]{adjustbox}
\usepackage{graphicx}
\usepackage[utf8]{inputenc}
\usepackage{multicol}
\usepackage{multirow}
\usepackage{makeidx}


\usepackage{hyperref}
\hypersetup{
    colorlinks,
    citecolor=black,
    filecolor=black,
    linkcolor=black,
    urlcolor=black
}

% \usepackage[backend=biber
%            % ,style=authoryear-icomp
%             ]{biblatex}
% \addbibresource{used_sources.bib}

% NLS support packages
\usepackage[T2A]{fontenc}
\usepackage[russian]{babel}
\usepackage{pscyr}

% Font selection
\usepackage{courier}
\usepackage{amssymb}

\setlength{\parindent}{0cm}
\setlength{\parskip}{0.2cm}

\newcommand\tab[1][1cm]{\hspace*{#1}}

% debug to see the frame borders
% from https://en.wikibooks.org/wiki/LaTeX/Page_Layout
% \usepackage{showframe}

% Indices & bibliography
%\usepackage{natbib}
\usepackage[titles]{tocloft}
\setcounter{tocdepth}{3}
\setcounter{secnumdepth}{5}
\makeindex

\renewcommand{\cftsecleader}{\cftdotfill{\cftdotsep}}

% change style of titles in \section{}
\usepackage{titlesec}
\titleformat{\section}[hang]{\Large\bfseries\center}{\thetitle.}{1em}{}
\titleformat{\subsection}[hang]{\large\normalfont\raggedright}{\thetitle.}{1em}{\underline}
\titleformat{\subsubsection}[hang]{\normalsize\normalfont\raggedright}{\thetitle.}{1pt}{}

% Packages for text layout in normal mode
% \usepackage[parfill]{parskip} % автоматом делает пустые линии между параграфами, там где они есть в тексте
% \usepackage{indentfirst} % indent even in first paragraph
\usepackage{setspace}    % controls space between lines
\setstretch{1} % space between lines
\setlength\parindent{0.9cm} % size of indent for every paragraph
\usepackage{csquotes}% превратить " " в красивые двойные кавычки
\MakeOuterQuote{"}


% this makes items spacing single-spaced in enumerations.
\newenvironment{my_enumerate}{
    \begin{enumerate}
        \setlength{\itemsep}{1pt}
        \setlength{\parskip}{0pt}
        \setlength{\parsep}{0pt}}{\end{enumerate}
}
\usepackage{pbox}

% configure eskd
\titleTop{
    {\Large ПРАВИТЕЛЬСТВО РОССИЙСКОЙ ФЕДЕРАЦИИ \\
        НАЦИОНАЛЬНЫЙ ИССЛЕДОВАТЕЛЬСКИЙ УНИВЕРСИТЕТ \\
        <<ВЫСШАЯ ШКОЛА ЭКОНОМИКИ>>} \\
    \vspace*{0.2cm}
    {Факультет компьютерных наук \\
        Департамент программнoй инженерии \\
    }
}
\titleDesignedBy{Студент группы БПИ 151 НИУ ВШЭ}{Куприянов К. И.}

\titleAgreedBy{
    \parbox[t]{7cm} {
        \centerline{Профессор департамента}
        \centerline{программной инженерии факультета}
        \centerline{компьютерных наук, канд. техн. наук}
}}{С.М. Авдошин}
\titleApprovedBy{
    \parbox[t]{7cm} {
        \centerline{Академический руководитель}
        \centerline{образовательной программы}
        \centerline{<<Программная инженерия>>}
        \centerline{профессор, канд. техн. наук}
}}{В. В. Шилов}
\titleName{Криптографические алгоритмы и протоколы для распределенных реестров}
\workTypeId{RU.17701729.506300 T3 01-1}

\titleSubname{Техническое задание}

\begin{document}
    \pagenumbering{arabic}

    \section{Аннотация}
    Технология блокчейн обычно ассоциируется с криптовалютой биткойн, потому что
биткойн - первая повсеместно используемая система, использующая блокчейн как
основу. По мере развития технологий число различных блокчейнов со множеством
способов их приложения резко возросло. Факт существования такого значительного
их количества можно объяснить тем, что при их реализации могут варьироваться
используемые криптографические алгоритмы и протоколы. В связи с этим возникла
проблема отсутствия систематически собранной и структурированной информации о
криптографических алгоритмах и протоколах в существующих распределенных
реестрах. Главной целью данной работы является сбор и обобщение известных и
распространенных на сегодняшний день криптографических алгоритмов и протоколов.
Предложен сравнительный анализ алгоритмов, используемых в блокчейнах, по общим
показателям. Также в качестве инструмента для разработчиков при создании
персонального распределенного реестра в образовательных целях разработана
библиотека на языке Python3.6, в которой собраны реализации проанализированных
алгоритмов и протоколов.\\

\textbf{Ключевые слова} --- блокчейн, биткоин, распределённый реест,
технология распределённого реестра, криптография, классификация, Python.


    \newpage
    \tableofcontents

    % --- add my custom headers ---

    \newpage
    \section{Введение}
    \subsection{Наименование программы}
Наименование программы на русском:
``Криптографические алгоритмы и протоколы для распределенных реестров''. \\
Наименование на английском:
``Cryptographic Algorithms and Protocols for Distributed Ledgers''. \\


\subsection{Краткая характеристика}
Программа предназначена для пользователей машин на семействе ОС GNU/Linux.
Цель работы --- создать удобное приложение для получения готовых кодов
алгоритмов и протоколов, рассмотренных в теоретической части работы.  Этот
библиотека будет служить ``инструментарием'' для программиста или любого
другого интересующегося криптографическими алгоритмами и протоколами, который
хочет узнать как работают современные распределённые реестры с рассмотренными
аспектами. Это позволит быстро получать необходимую техническую информацию,
которую с трудом можно найти в общем доступе. Программа должна предоставлять не
только генерацию кода, но и дружелюбный интерфейс командной строки, в которой
форматирование и подсветка не будут сбивать с толку неподготовленного
пользователя. Должен быть реализован алгоритм чтения, обработки и валидации
конфигурационного файла на языке Yaml.\\

Главной чертой данного приложения является его лёгкая, быстрая
масштабируемость, модульность программного кода, а так же вся теоретическая
база, которая лежит в основе информационной модели.


    % \newpage
    % \section{Основания для разработки}
    % \subsection{Документ, на основании которого ведется разработка}
Основной приказ декана ФКН И.В. Аржанцева о назначении тем Курсовых Работ
\textnumero 2.3-02/1212-01 от 12.12.2017

\subsection{Наименование темы разработки}
Наименование темы: ``Клиент-серверное Android-Приложение для Управления Скидками в Розничных Сетях''. \\
Наименование темы на английском: ``The Client-Server Android Application for Managing the Products' Discounts in Retail Networks''.


    % \newpage
    % \section{Назначение разработки}
    % \subsection{Функциональное назначение}
К функциональным возможностям программы относятся:
выгрузка кодов алгоритмов из источников по путям из хранилища, установка
алгоритмов в систему без прав суперпользователя, и генерация кода готового
проекта (реализации блокчейна) по указанной директории. Код реализации
блокчейна должен использовать алгоритмы, выбранные пользователем.

\subsection{Эскплуатационное назначение}
Программа предназначена для запуска на машинах операционных систем GNU/Linux.
Продукт является свободным ПО, что позволяет беспрепятственно скачивать,
изучать исходные файлы, использовать их для своих нужд, изменять и
распространять как приобретённые копии, так и изменённые (лицензия GNU GPL v3
\cite{gnu}).
Программа даёт пользователю возможность менять свой выбор ``на лету'' и
изменять список, ещё не приступив к генерации кодов; так же есть возможность
посмотреть справочную информацию по выбранным параметрам блокчейна.

% EOF



    \newpage
    \section{Требования к программе}
    \subsection{Требования к функциональным характеристикам}
\subsubsection{Требования к составу выполняемых функций. Клиентская часть (Android приложение).}
\begin{my_enumerate}
\item Возможность просмотра списка доступных магазинов с акционными товарами
\item Представление текущих акций для конкретного магазина:
    \begin{my_enumerate}
    \item В виде общего списка
    \item По категориям
    \end{my_enumerate}
\item Постепенная загрузка товаров магазинов (по страницам) для экономии трафика и меньшей нагрузкой на мобильное устройство
\item Регистрация через мобильное приложение
\item Вход в аккаунт через мобильное приложение
\item Возможность смены аккаунта
\item Возможность создания списков покупок c разными названиями
\item Возможность удаления списка покупок
\item Возможность добавления товара в список покупок
\item Возможность удаления товара из списка покупок
\item Возможность добавления в список покупок пользовательских товаров, котороых нет в магазине (см. терминологию)
\item Возможность просмотра подобранных программой товаров согласно запросу пользователя
\item Добавление подобранных товаров в список покупок
\item Предварительное отображение элементов каждого списка покупок до их открытия
\item Отображение всплывающих подсказок при долгом нажатии на элементы управления button (кнопка)
\item При отсутствии интернет-соединения перенаправление в настройки сети для последующего включения интернета
\item Отображение индикатора процесса загрузки данных с сервера
\item Реализация обучающего фрагмента в разделе help (см. терминологию), содержащего руководство пользователя по управлению программой
\end{my_enumerate}

\subsubsection{Требования к составу выполняемых функций. Серверная часть.}
\begin{my_enumerate}
\item Crawling веб-страниц для сбора актуальной информации об акционных товарах
\item Добавление товаров в базу данных посредством отправки запросов REST API (REST API и база данных реализованы напарником)
\item Запись акционных товаров во всех магазинах в JSON файл
\item Email уведомления администраторам сервиса об ошибках и неполадках в работе сервера
\item Ежедневное обновление акций для поддержания актуальности
% \item Реализация панели администратора для управления web-crawler'ом
\end{my_enumerate}

\subsection{Требования к временным характеристикам}
При скорости интернет соединения 30Мбит/с:
\begin{my_enumerate}
\item Загрузка одной страницы с товарами -- не более 6 секунд
\item Отправка списка покупок на сервер -- не более 3 секунд
\end{my_enumerate}

\subsection{Требования к интерфейсу}
\begin{my_enumerate}
\item Совместимость с графической подсистемой ОС Android {\textregistered}
\item Язык интерфейса: русский и английский, в зависимости от выбранного в настройках мобильного устройства
\item Оформление программы в стиле соответствующему guideline от Google: \url{http://material.io/guidelines/style/color.html}
\item Интуитивная ясность конечному пользователю без наличия специального или профессионального образования
\item Полоса загрузки в центре экрана для индикации состояния скачивания данных с сервера
\end{my_enumerate}

\subsection{Требования к надежности}
\subsubsection{Обеспечение устойчивого функционирования программы}

Для надежной работы программы требуется исполнение следующих требований:
\begin{my_enumerate}
\item Обеспечение поддержания заряда аккумуляторной батареи устройства на
уровне не ниже 30\%, иначе обеспечить бесперебойную подзарядку оборудования
\item Обеспечение использования лицензионного программного обеспечения
\item Обеспечение защиты операционной системы и технических средств от
вредоносного воздействия шпионских программ, компьютерных вирусов и сетевых
червей
\item Обеспечение своевременного обновления программных составляющих мобильного устройства
\item При изменении дизайна веб-сайта магазина, администратор приложения оперативно
исправляет соответствующие селекторы для кроулера
\item Раз в сутки производить бэкап всех баз данных
\end{my_enumerate}


    % \newpage
    % \section{Требования к программной документации}
    % \subsection{Предварительный состав программной документации}
\begin{my_enumerate}
    \item ``Криптографические алгоритмы и протоколы для распределенных реестров. Техническое 
    задание''
    \item ``Криптографические алгоритмы и протоколы для распределенных реестров. 
    Пояснительная записка''
    \item ``Криптографические алгоритмы и протоколы для распределенных реестров. 
    Руководство оператора''
    \item ``Криптографические алгоритмы и протоколы для распределенных реестров. Программа и 
    методика испытаний''
    \item ``Криптографические алгоритмы и протоколы для распределенных реестров. Текст 
    программы''
\end{my_enumerate}



    % \newpage
    % \section{Технико-экономические показатели}
    % \subsection{Оринтировочная экономическая эффективность}
Оринтировочная экономическая эффективность не рассчитывается.

\subsection{Экономические преимущества разработки}
Существующим аналогом данного продукта является пользовательская онлайн-таблица
PrivacyCoinMatrix, собранная энтузиастами в сообществе (\cite{reddit}). В силу
того, что данное приложение имеет интерфейс командной строки, оно легковеснне и
не требует запуска тяжёлого браузера для просмотра массивных таблиц. Всё, что
нужно сделать это установить пакет. Приложение даёт возможность интерактивного
последовательного выбора алгоритмов и протоколов, в то время как в таблице
необходимо пользоваться ``ручным'' поиском по тексту страницы. Исходя из этого,
можно заключить, что данное приложение выступает с положительных сторон
относительно аналога.


    % \newpage
    % \section{Стадии и этапы разработки}
    % \subsection{Необходимые стадии разработки}
\subsubsection{Техническое задание}
Этапы разработки:
\begin{my_enumerate}
    \item Обоснование необходимости разработки программы
        \begin{my_enumerate}
            \item постановка задачи
            \item сбор исходных материалов
            \item обоснование необходимости проведения научно-исследовательских работ
        \end{my_enumerate}
    \item Научно-исследовательские работы
        \begin{my_enumerate}
            \item определение структуры входных и выходных данных;
            \item предварительный выбор методов решения задач;
            \item определение требований к техническим средствам.
        \end{my_enumerate}
    \item Разработка и утверждение технического задания
        \begin{my_enumerate}
            \item определение требований к программе
            \item определение стадий, этапов и сроков разработки программы и документациик ней
        \end{my_enumerate}
\end{my_enumerate}


\subsubsection{Технический проект}
Этапы разработки:
\begin{my_enumerate}
    \item Разработка технического проекта
    \begin{my_enumerate}
        \item разработка технического проекта
        \item разработка структуры программы
    \end{my_enumerate}
    \item Утверждение технического проекта
    \begin{my_enumerate}
        \item разработка плана мероприятий по разработке программы
        \item разработка пояснительной записки
    \end{my_enumerate}
\end{my_enumerate}


\subsubsection{Рабочий проект}
\begin{my_enumerate}
    \item Разработка программы
        \begin{my_enumerate}
            \item программирование и отладка программы
            \item создание пакета инсталляции программы
        \end{my_enumerate}
    \item Разработка программной документации
        \begin{my_enumerate}
            \item разработка программных документов в соответствии с
                требованиями ГОСТ 19.101--77
        \end{my_enumerate}
     \item Испытания программы
           \begin{my_enumerate}
             \item разработка, согласование и утверждение программы и методики испытаний
            \item корректировка программы и программной документации по результатами испытаний
         \end{my_enumerate}
\end{my_enumerate}

\subsection{Сроки работ и исполнители}

Приложение должно быть разработано к 1 апреля 2018 года, студентом группы БПИ151
Куприяновым Кириллом.


% EOF


    % \newpage
    % \section{Порядок контроля и приемки}
    % Контроль и приемка разработки осуществляются в соответствии с документом: 
``Криптографические алгоритмы и протоколы для распределенных реестров''. 
Программа и методика испытаний''. \\
Испытания проводятся поэтапно, друг за другом, в следующем порядке:
\begin{my_enumerate}
	\item Испытание выполнения требований к программной документации
	\item Испытание выполнения требований к консольному интерфейсу и оформлению программы
	\item Испытание выполнения требований к функциональным характеристикам программы, надежности и корректности ее работы
	\item Испытание выполнения требований к временным характеристикам
\end{my_enumerate}


    % \newpage
    % \section{Приложение 1. Терминология}
    % \subsection{Терминология}
\begin{description}
    \item[Распределённый реестр (Distributed Ledger)] ---
        В примитивной своей реализации это распределённая база данных между
        сетевыми узлами или вычислительными устройствами.
        Каждый из узлов может получать данные других, при этом храня полную
        копию реестра. Обновления этих узлов происходят независимо друг от
        друга.

    \item[Блокчейн] ---
        Постоянно растущий список записей, называемых блоками, которые связаны
        и защищены с помощью криптографии. Он копируется его пользователями и
        устойчив к модификации. Машина с рабочей копией называется узлом.

    \item[DAG] ---
        Направленный ациклический граф. Это ориентированный граф с данными,
        использующий топологическую сортировку (от ранних узлов к более поздним).

    \item[Биткоин (Bitcoin)] ---
        Электронная пиринговая платёжная система, используемая в качестве
        финансовой единицы (криптовалюты) одноимённую сущность. Создателем
        биткоина выступает некто Satoshi Nakomoto \cite{Nakamoto2008}.

    \item[Эфириум (Ethereum)] ---
        Открытая, общедоступная, вторая по популярности, распределенная
        вычислительная платформа на основе технологии блокчейн и операционная
        система с функциональностью смарт-контрактов
        \cite{VitalikButerin2015}

    \item[Алгоритм консенсуса] ---
        Набор математических операций, которые необходимо выполнять для
        поддержания консистентности всей сети.
\end{description}



    \newpage
    \section{Приложение 2. Список используемой литературы}
    % \input{used_literature}

    % Index
    \newpage
    \eskdListOfChanges

\end{document}


% EOF
