Автообновление настроено на отдельном сервере по расписанию. Конфигурация
сервера представляет из себя VPS виртуальную машину, на которой установлена OS
\emph{Ubuntu 16.04 LTS} с версией ядра \emph{4.4.0-148-generic}.

Расписание автообновления сконфигурировано при помощи стандартной UNIX утилиты
\textbf{cron}.  Конфигурация \textbf{crontab} представлена в листинге
\ref{lst:cron}. Происходят там следующие процессы:

\begin{itemize}
    \item Каждый день в 20:00 запускается скрипт {\small updater.py}, который получает актуальные изменения алгоритмов в локальную копию на серверную ФС;
    \item Каждый день в 21:00 запускается скрипт {\small pusher.py}, которые полученные изменения загружает в публичный репозиторий, делая обновлённые и актуальные данные доступными всем;
    \item Логи обновления и загрузки пишутся в {\small updater.log} и {\small pusher.log} соответственно.
\end{itemize}

\begin{center}
\begin{lstlisting}
        00 20 * * * echo $(date) >> /home/coldmind/gsl/updater.log && cd /home/coldmind/gsl/src && ./updater.py >> /home/coldmind/gsl/updater.log 2>&1
        00 21 * * * echo $(date) >> /home/coldmind/gsl/pusher.log && cd /home/coldmind/gsl && ./src/pusher.sh >> /home/coldmind/gsl/pusher.log 2>&1
\end{lstlisting}\label{lst:cron}
    Листинг \ref{lst:cron}: Конфигурация {\small crontab -l}
\end{center}

Код скрипта обновления {\small updater.py} представлен в листинге
\ref{lst:updater}. Он использует реализованное key-value хранилище (REF) для
получения ссылок, по которым возможно обновление загруженных алгоритмов, и для
каждой ссылки и пути записи в ФС (тоже получает из key-value хранилища),
запускает скрипт {\small pull\_single.sh} (Листинг \ref{lst:pull_single}).

\paragraph{updater}

\begin{center}
\begin{lstlisting}
#!/usr/bin/env python3.6

import subprocess
from technologies import _kv_, get
toinstall = _kv_.toinstall
update_links = _kv_.update_links

for alg, src in update_links.items():
    p = subprocess.popen(['bash', 'pull_single.sh', f'{get("toinstall", alg)}', f'{src}'], stdout=subprocess.pipe)
    (result, error) = p.communicate()
    print(result.decode())
\end{lstlisting}\label{lst:updater}
    листинг \ref{lst:updater}: Скрипт обновления всех алгоритмов со внешних источников
\end{center}

Скрипт {\small pull\_single.sh} (Листинг \ref{lst:pull_single}) представляет из
себя поход по ссылке во внешний источник, выгрузку данных оттуда, и замещение
новыми данными старые.

\begin{center}
\begin{lstlisting}
#!/usr/bin/env bash

if cd $1;
then
    cd ..
    name1=$(echo $1 | rev | cut -d/ -f1 | rev)
    name2=$(echo $2 | rev | cut -d/ -f1 | rev)
    if [[ $name1 == $name2 ]]
    then
        rm -rf $name1
        git clone $2
        cd $name2
        rm -rf .git*
    fi
fi
\end{lstlisting}\label{lst:pull_single}
    листинг \ref{lst:pull_single}: Скрипт обновления единичного алгоритма
\end{center}


\paragraph{pusher}

Скрипт {\small pusher.py} используется для загрузки полученных изменений в сеть
для общего доступа. Представлен в листинге \ref{lst:pusher}. При неизвестной
ошибке в процессе обновления или некорректной работе обновлённых алгоритмов,
подключенный к репозиторию сервис Continuous Integration не позволит вступить
изменениям в силу, вследствие чего, пользователи будут защищены от нерабочего
кода.

\begin{center}
\begin{lstlisting}
#!/usr/bin/env bash

curr_date=$(date +%d_%m_%Y)

git add .
git commit -m "Update algorithms: $curr_date."
git push


# EOF
\end{lstlisting}\label{lst:pusher}
    листинг \ref{lst:pusher}: Скрипт для загрузки обновлённых данных
\end{center}
