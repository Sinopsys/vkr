\tab[0.88cm]В последние годы наблюдается усиленный рост интереса общества к криптовалютам,
блокчейнам и распределённым реестрам.
Последние десять лет показали, что прогресс в области изучения распределенных
реестров и криптовалюты огромен по сравнению с объёмом прогресса в любой
другой областью в прошлом.
Одним из результатов растущего интереса --- это тысячи исследований,
проведенных учеными и исследователями.

Эти исследования положили начало
созданию сотен криптовалют и других видов применений распределенных регистров.
Некоторые криптовалюты повторяют друг друга в плане алгоритмов (хэширования,
других крипто-алгоритмов) и протоколов, но их количество огромное. Относительно
мало исследований было проведено на используемые алгоритмы и еще меньше на
протоколы. Обобщённость многих статей на эти темы проблематична. В то время как
большинство веб-сайтов предоставляют ненаучные бизнес-объяснения, есть одно
исследование \cite{TimSwanson2014}, в котором присутствует некоторая
агрегация информации об алгоритмах и протоколах в распределенных реестрах.

Однако, во-первых, остаётся ряд неотъемлимых технических вопросов о
классификации алгоритмов и протоколов для распределенных реестров, а во-вторых,
данное исследование имело место быть в 2014 году, что делает информацию в нем
устаревшей.  Появляются новые алгоритмы и современные приложения старых, новые
криптовалюты и их протоколы публикуются и запускаются каждый год, и в в
настоящий момент есть много возможностей для расширения в этой области.

Данная работа сфокусирована на исследовании использования современных
алгоритмов и протоколов в распределенных реестрах, актуализации их
классификации.

Цель состоит в том, чтобы расширить существующую классификацию
новыми алгоритмами и протоколами, фокусируясь на ширине обхвата современных
технологий, а не на их количесве. Обновлённая классификация отражает
фактическое состояние алгоритмов и протоколов в распределённых реестрах на
вторую четверть 2019 года. Помимо этого, важной частью работы является сбор
всех описанных реализованных алгоритмов в программной библиотеке на языке
Python3.6 для свободного публичного использования. Этот библиотека будет
служить ``инструментарием'' для программиста, который хочет узнать как работают
современные распределённые реестры с рассмотренными алгоритмами.

\textbf{Научная новизна работы} заключается в том, что была разработана новая
актуальная классификация алгоритмов и протоколов для распределённых ресстров (и
самих реестров), тем самым дополнив существующую (но устаревшую) на сегодняшней
день классификацию 2014-го года.

\textbf{Практическая значимость работы} заключается в том, что разработанная
программа способна автоматизировать процесс программирования распределённых
реестров и может быть использована организациями и физическими лицами для
создания приложений с использованием готовой архитектуры блокчейна с
уникальными (выбранными пользователем) алгоритмами.
