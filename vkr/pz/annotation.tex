Технология блокчейн обычно ассоциируется с криптовалютой биткойн, потому что
биткойн - первая повсеместно используемая система, использующая блокчейн как
основу. По мере развития технологий число различных блокчейнов со множеством
способов их приложения резко возросло. Факт существования такого значительного
их количества можно объяснить тем, что при их реализации могут варьироваться
используемые криптографические алгоритмы и протоколы. В связи с этим возникла
проблема отсутствия систематически собранной и структурированной информации о
криптографических алгоритмах и протоколах в существующих распределенных
реестрах. Целью данной работы является сбор и классификация по использованию в
реестрах известных и распространенных на сегодняшний день криптографических
алгоритмов и протоколов. А в качестве программной составляющей проекта ---
инструмент, позволяющий создать персональный распределенный реестр в
образовательных или прикладных целях. Это библиотека на языке Python3.6, в
которой реализована имплементация блокчейна, а так же собраны реализации
рассмотренных алгоритмов.\\

\textbf{Ключевые слова} --- блокчейн, биткоин, распределённый реест,
технология распределённого реестра, криптография, классификация, Python.
