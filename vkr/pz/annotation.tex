Технология блокчейн обычно ассоциируется с криптовалютой биткойн, потому что
биткойн - первая повсеместно используемая система, использующая блокчейн как
основу. По мере развития технологий число различных блокчейнов со множеством
способов их приложения резко возросло. Факт существования такого значительного
их количества можно объяснить тем, что при их реализации могут варьироваться
используемые криптографические алгоритмы и протоколы. В связи с этим возникла
проблема отсутствия систематически собранной и структурированной информации о
криптографических алгоритмах и протоколах в существующих распределенных
реестрах. Главной целью данной работы является сбор и обобщение известных и
распространенных на сегодняшний день криптографических алгоритмов и протоколов.
Предложен сравнительный анализ алгоритмов, используемых в блокчейнах, по общим
показателям. Также в качестве инструмента для разработчиков при создании
персонального распределенного реестра в образовательных целях разработана
библиотека на языке Python3.6, в которой собраны реализации проанализированных
алгоритмов и протоколов.\\

\textbf{Ключевые слова} --- блокчейн, биткоин, распределённый реест,
технология распределённого реестра, криптография, классификация, Python.
