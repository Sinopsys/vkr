К функциональным требованиям приложения компоновщик относятся:

\begin{my_enumerate}
    \item Возможность вывода на консоль вариантов выбора по категориям:
        \begin{my_enumerate}
                \item Отображение возможных структур реестра
                \item Отображение возможных типов открытости реестра
                \item Отображение возможных алгоритмов консенсуса
                \item Отображение возможных алгоритмов хэширования
                \item Отображение возможных алгоритмов генерации случайных чисел
                \item Отображение возможных алгоритмов цифровой подписи
        \end{my_enumerate}
    \item Генерирование значений для выбора ``по умолчанию''
    \item Возможность записать выбора пользователя
    % \item Возможность поиска в хранилище ссылок для конкретных алгоритмов
    % \item Возможность загрузки из общедоступных источников исходных кодов алгоритмов
    \item Возможность установки загруженных алгоритмов на ФС машины без исключительных root прав
    \item Возможность генерировать код по указанной директории
    \item Возможность замера времени работы выбранных алгоритмов
    \item Возможность просмотра справочной информации по остальным параметрам реестра
    \item Вывод информации в цвете, обозначающий степень поддержки программой алгоритма
\end{my_enumerate}


К функциональным требованиям приложения реализация блокчейна ({\small wallet.py})относятся:
\begin{enumerate}
    \item Возможность генерации ``адреса кошелька'' --- пары приватный + публичный ключ
    \item Использование в качестве алгоритма цифровой подписи выбранный пользователем
    \item Использование в качестве алгоритма хэширования выбранный пользователем
    \item Возможность записи данных ``адреса кошелька'' на ФС машины
    \item Возможность отправки от одного пользователя другому условное
          количество условной криптовалюты
    \item Возможность получения всей цепочки транзакций, которые были проведены
          за текущую сессию путём вызова API функции майнера
\end{enumerate}

К функциональным требованиям приложения реализация блокчейна ({\small miner.py})относятся:
\begin{enumerate}
    \item Возможность принимать json сообщения по http протоколу
    \item Использование в качестве алгоритма цифровой подписи выбранный пользователем
    \item Использование в качестве алгоритма хэширования выбранный пользователем
    \item Возможность выполнения proof-of-work алгоритма
    \item Возможность добавления блока в цепочку
\end{enumerate}
