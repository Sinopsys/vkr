\subsubsection{Общая структура проектного решения}
Проект состоит из двух подпроектов. Первый отвечает за интерактивное
взаимодействие с пользователем и генерации кода для второго проекта с
использованием указанных пользователем алгоритмов (далее --- компоновщик).
Второй проект это имплементация блокчейна (далее --- реализация блокчейна).
Проект содержит код для кошелька ({\small wallet.py}) и майнера ({\small
miner.py}) с определённой функциональностью. Код второго проекта структурирован
для удовлетворения нужд использования указанных пользователем методов. Методы и
классы генерируются at-runtime первого приложения.

\begin{itemize}
    \item Язык программирования --- Python версии 3.6.5
    \item Тип: консольная утилита
    \item Протокол обмена данными между компонентами: http + json
    \item База данных: key-value хранилище
    \item Запуск скриптов обновления: UNIX cron
    \item Continuous integration: Shippable
    \item License: GNU GPL v3
\end{itemize}

\subsubsection{Архитектура компоновщика}
Компоновщик --- часть проекта, автоматизирующая процесс программирования и
позволяющяя тем самым создавать готовые решения. Решением может быть рабочий
код блокчейна с использованием 24 вариаций алгоритмов.
\subsubsection{Порядок работы с компоновщиком}
\subsubsection{Архитектура реализации блокчейна}


\subsubsection{Язык программирования}
Выбор пал
