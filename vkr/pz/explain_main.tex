\documentclass[
%a4paper,12pt
encoding=utf8
]{twoeskd}

% Packages required by doxygen
\usepackage{fixltx2e}
\usepackage{calc}
\usepackage{doxygen}
\usepackage[export]{adjustbox} % also loads graphicx
\usepackage{graphicx}
\usepackage[utf8]{inputenc}
\usepackage{makeidx}
\usepackage{multicol}
\usepackage{multirow}
\PassOptionsToPackage{warn}{textcomp}
\usepackage{textcomp}
\usepackage[nointegrals]{wasysym}
\usepackage{qtree}
\usepackage{pgfplots}
    \pgfplotsset{
        compat=1.12,
    }

% NLS support packages
\usepackage[T2A]{fontenc}
\usepackage[russian]{babel}

% Font selection
\usepackage{courier}
\usepackage{amssymb}
\usepackage{sectsty}
\usepackage{fancyvrb}
\renewcommand{\familydefault}{\sfdefault}
\newcommand{\+}{\discretionary{\mbox{\scriptsize$\hookleftarrow$}}{}{}}

% Page & text layout
\usepackage{geometry}
\tolerance=750
\hfuzz=15pt
\hbadness=750
\setlength{\emergencystretch}{15pt}
\setlength{\parindent}{0cm}
\setlength{\parskip}{0.2cm}
\makeatletter
\makeatother

% Headers & footers
% \usepackage{fancyhdr}
% \renewcommand{\sectionmark}[1]{%
%   \markright{\thesection\ #1}%
% }
\usepackage[backend=biber
           ,style=gost-numeric
            ]{biblatex}
\addbibresource{../library.bib}

% Indices & bibliography
% \usepackage{natbib}
\usepackage[titles]{tocloft}

\usepackage{titletoc}
\setcounter{tocdepth}{3}
\setcounter{secnumdepth}{5}
\makeindex

\newcommand\tab[1][1cm]{\hspace*{#1}}
\usepackage{hyperref}
\hypersetup{
    colorlinks,
    citecolor=black,
    filecolor=black,
    linkcolor=black,
    urlcolor=black
}
% Custom commands
\newcommand{\clearemptydoublepage}{%
  \newpage{\pagestyle{empty}\cleardoublepage}%
}
\renewcommand{\DoxyLabelFont}{%
  \fontseries{bc}\selectfont%
}
\newcommand\degr{$^\circ$}

%TODO DELETE !!!
% \renewcommand{\cite}{}

% Custom packages
\usepackage{pdfpages}


\setlength{\parindent}{0cm}
\setlength{\parskip}{0.2cm}

% debug to see the frame borders
% from https://en.wikibooks.org/wiki/LaTeX/Page_Layout
% \usepackage{showframe}

% change style of titles in \section{}
\usepackage{titlesec}
\titleformat{\section}[hang]{\huge\bfseries\center}{\thetitle.}{1em}{}
\titleformat{\subsection}[hang]{\Large\raggedright}{\thetitle.}{1em}{\underline}
\titleformat{\subsubsection}[hang]{\large\raggedright}{\thetitle.}{1pt}{}

% Packages for text layout in normal mode
% \usepackage[parfill]{parskip} % автоматом делает пустые линии между параграфами, там где они есть в тексте
% \usepackage{indentfirst} % indent even in first paragraph
\usepackage{setspace}    % controls space between lines
\setstretch{1} % space between lines
\setlength\parindent{0.9cm} % size of indent for every paragraph
\usepackage{csquotes}% превратить " " в красивые двойные кавычки
\MakeOuterQuote{"}

\usepackage{listings}
\usepackage{color}

\definecolor{dkgreen}{rgb}{0,0.6,0}
\definecolor{gray}{rgb}{0.5,0.5,0.5}
\definecolor{mauve}{rgb}{0.58,0,0.82}

\lstset{frame=tb,
  language=Python,
  aboveskip=3mm,
  belowskip=3mm,
  showstringspaces=false,
  columns=flexible,
  basicstyle={\small\ttfamily},
  numbers=none,
  numberstyle=\tiny\color{gray},
  keywordstyle=\color{blue},
  commentstyle=\color{dkgreen},
  stringstyle=\color{mauve},
  breaklines=true,
  breakatwhitespace=true,
  tabsize=3
}

% this makes items spacing single-spaced in enumerations.
\newenvironment{my_enumerate}{
    \begin{enumerate}
        \setlength{\itemsep}{1pt}
        \setlength{\parskip}{0pt}
        \setlength{\parsep}{0pt}}{\end{enumerate}
}
\usepackage{pbox}


\setcounter{secnumdepth}{4}

\titleformat{\paragraph}
{\normalfont\normalsize}{\theparagraph}{0.4em}{}
\titlespacing*{\paragraph}
{0pt}{3.25ex plus 1ex minus .2ex}{1.5ex plus .2ex}

% configure eskd
\titleTop{
    {\Large ПРАВИТЕЛЬСТВО РОССИЙСКОЙ ФЕДЕРАЦИИ \\
        НАЦИОНАЛЬНЫЙ ИССЛЕДОВАТЕЛЬСКИЙ УНИВЕРСИТЕТ \\
        <<ВЫСШАЯ ШКОЛА ЭКОНОМИКИ>>} \\
    \vspace*{0.2cm}
    {Факультет компьютерных наук \\
        Департамент программнoй инженерии \\
    }
}
\titleDesignedBy{Студент группы БПИ 151 НИУ ВШЭ}{Куприянов К. И.}

\titleAgreedBy{
    \parbox[t]{7cm} {
        \centerline{Профессор департамента}
        \centerline{программной инженерии факультета}
        \centerline{компьютерных наук, канд. техн. наук}
}}{С.М. Авдошин}
\titleApprovedBy{
    \parbox[t]{7cm} {
        \centerline{Академический руководитель}
        \centerline{образовательной программы}
        \centerline{<<Программная инженерия>>}
        \centerline{профессор, канд. техн. наук}
}}{В. В. Шилов}
\titleName{Выпускная квалификационная работа\\
по теме \underline{Криптографические алгоритмы и протоколы для распределенных реестров}}
\workTypeId{\color{white} . }
\titleSubname{по направлению подготовки 09.03.04 <<Программная инженерия>>}


%===== C O N T E N T S =====
\begin{document}

\titlecontents{section}[0em]
{\vskip 0.5ex}%
{}% numbered sections formattin
{}% unnumbered sections formatting
{}%

% Titlepage & ToC

\thispagestyle{empty}
\pagenumbering{arabic}
\paragraph*{\huge Реферат\\\\}
Технология блокчейн обычно ассоциируется с криптовалютой биткойн, потому что
биткойн - первая повсеместно используемая система, использующая блокчейн как
основу. По мере развития технологий число различных блокчейнов со множеством
способов их приложения резко возросло. Факт существования такого значительного
их количества можно объяснить тем, что при их реализации могут варьироваться
используемые криптографические алгоритмы и протоколы. В связи с этим возникла
проблема отсутствия систематически собранной и структурированной информации о
криптографических алгоритмах и протоколах в существующих распределенных
реестрах. Главной целью данной работы является сбор и обобщение известных и
распространенных на сегодняшний день криптографических алгоритмов и протоколов.
Предложен сравнительный анализ алгоритмов, используемых в блокчейнах, по общим
показателям. Также в качестве инструмента для разработчиков при создании
персонального распределенного реестра в образовательных целях разработана
библиотека на языке Python3.6, в которой собраны реализации проанализированных
алгоритмов и протоколов.\\

\textbf{Ключевые слова} --- блокчейн, биткоин, распределённый реест,
технология распределённого реестра, криптография, классификация, Python.


\newpage

\paragraph*{\huge Abstract\\\\}
The Blockchain technology is typically associated with Bitcoin, because it was
the first system which has been distributed using the Blockchain technology.
As the technologies evolved, the number of various blockchains with different
kinds of applications had been drastically risen. A huge amount of blockchains
can be explained by various cryptographic algorithms and protocols usage in
them. It brought a problem of the absence of systematically gathered and
structured information about cryptographic algorithms and protocols in existing
distributed ledgers. The main objective of this work is to generalise all known
common cryptographic algorithms and protocols, which are being used nowadays.
The algorithms used in blockchains are going to be classified by common
metrics: time complexity, space complexity, and the resistance to hacking. It
is also intended to bring a programming library, where analyzed algorithms and
protocols are gathered in one place. The library would serve as a toolbox for
developers when creating a personal distributed ledger.  Structured information
and an accompanying programming toolbox will preliminary cause positive effect
on approaching extension in this area.

% \pagenumbering{arabic}
\newpage
\tableofcontents
% \pagenumbering{arabic}

% --- add my custom headers ---

\newpage
\section{Определения}
\subsection{Терминология}
\begin{description}
    \item[Распределённый реестр (Distributed Ledger)] ---
        В примитивной своей реализации это распределённая база данных между
        сетевыми узлами или вычислительными устройствами.
        Каждый из узлов может получать данные других, при этом храня полную
        копию реестра. Обновления этих узлов происходят независимо друг от
        друга.

    \item[Блокчейн] ---
        Постоянно растущий список записей, называемых блоками, которые связаны
        и защищены с помощью криптографии. Он копируется его пользователями и
        устойчив к модификации. Машина с рабочей копией называется узлом.

    \item[DAG] ---
        Направленный ациклический граф. Это ориентированный граф с данными,
        использующий топологическую сортировку (от ранних узлов к более поздним).

    \item[Биткоин (Bitcoin)] ---
        Электронная пиринговая платёжная система, используемая в качестве
        финансовой единицы (криптовалюты) одноимённую сущность. Создателем
        биткоина выступает некто Satoshi Nakomoto \cite{Nakamoto2008}.

    \item[Эфириум (Ethereum)] ---
        Открытая, общедоступная, вторая по популярности, распределенная
        вычислительная платформа на основе технологии блокчейн и операционная
        система с функциональностью смарт-контрактов
        \cite{VitalikButerin2015}

    \item[Алгоритм консенсуса] ---
        Набор математических операций, которые необходимо выполнять для
        поддержания консистентности всей сети.
\end{description}



\newpage
\section{Введение}
\subsection{Наименование программы}
Наименование программы на русском:
``Криптографические алгоритмы и протоколы для распределенных реестров''. \\
Наименование на английском:
``Cryptographic Algorithms and Protocols for Distributed Ledgers''. \\


\subsection{Краткая характеристика}
Программа предназначена для пользователей машин на семействе ОС GNU/Linux.
Цель работы --- создать удобное приложение для получения готовых кодов
алгоритмов и протоколов, рассмотренных в теоретической части работы.  Этот
библиотека будет служить ``инструментарием'' для программиста или любого
другого интересующегося криптографическими алгоритмами и протоколами, который
хочет узнать как работают современные распределённые реестры с рассмотренными
аспектами. Это позволит быстро получать необходимую техническую информацию,
которую с трудом можно найти в общем доступе. Программа должна предоставлять не
только генерацию кода, но и дружелюбный интерфейс командной строки, в которой
форматирование и подсветка не будут сбивать с толку неподготовленного
пользователя. Должен быть реализован алгоритм чтения, обработки и валидации
конфигурационного файла на языке Yaml.\\

Главной чертой данного приложения является его лёгкая, быстрая
масштабируемость, модульность программного кода, а так же вся теоретическая
база, которая лежит в основе информационной модели.


\newpage
\section{Глава 1. Обзор распределённых реестров, источников и решений}
В данной главе описаны возможные технологии распределённых реестров. Под
технологиями понимаются алгоритмы, протоколы, а так же общая структура реестра.
Распределённые реестры делятся на открытые (Public), закрытые (Private),
эксклюзивные (Permissioned), и инклюзивные (Permissionless).
По структуре реализации распределённые реестры делятся на 2 вида: блокчейны и
направленные ациклические графы (DAGs). Алгоритмы, представленные во всех типах
являются неотъемлимой их частью. Это алгоритмы хэширования, электронных
подписей, генерации случайных чисел. Специфические для конкретных реестров
алгоритмы, обеспечивающие должный уровень безопасности/анонимности, например
Ring signatures, Coinjoin, Coinshuffle, stealth addresses, MimbleWimble, etc.
будут рассмотрены в Главе 2. Помимо алгоритмов, будут рассмотрены и различные
протоколы консенсуса, (общей сложностью более 20 штук), обеспечивающие защиту и
надёжность транзакций в открытых блокчейнах.

% TODO
\subsection{Структура реестров}
\subsubsection{Блокчейны}
\subsubsection{Направленные ациклические графы (DAGs)}


\subsection{Открытые и закрытые реестры}
\subsubsection{Закрытые (private) реестры}
Не смотря на то, что сама рассматриваемая технология является распределённой,
элементы централизованности присутствуют в таких вариантах, как закрытые [X] и
эксклюзивные [X] реестры. Закрытым реестр может являться благодаря первичному
блоку (в случае блокчейна), который будет использоваться. Любой узел может
присоединиться к частному реестру, если он знает адрес начального узла
для синхронизации и идентификатор сети. Этот узел может выполнять
любые действия в частном реестре майнить, совершать транзакции, заключать
контракты, и т.д.

Такие реестры зачастую используюся фирмами, банками и т.д. для организации
внутренних операций обмена и регистрирования.

\subsubsection{Открытые (public) реестры}
В открытых реестрах нет контроллирующей стороны, как это реализовано в закрытых реестрах.

\subsection{Разрешённые реестры}
\subsubsection{Эксклюзивные (permissioned) реестры}
\subsubsection{Инклюзивные (permissionless) реестры}


\subsection{Алгоритмы}
\subsubsection{SHA-256}
\subsubsection{ECSDA}


\subsection{Особые алгоритмы}
\subsubsection{Ring Signatures}
\subsubsection{Stealth Addressess}
\subsubsection{Coinjoin, coinshuffle}
\subsubsection{Sigma Protocol}
\subsubsection{Zerocash}
\subsubsection{CT, RingCT}
\subsubsection{MimbleWimble}
\subsubsection{Zexe}


\subsection{Протоколы консенсуса}
\subsubsection{Proof of Work (PoW)}
Алгоритм доказательства работы (Proof of Work) используется в открытых реестрах
для создания новых блоков. Алгоритм хорошо проявляет себя для доказательства
легитимности транзакции и предотвращения ``двойной траты''.  Выполняя это
доказательство, майнеры вознаграждаются процентом, соответствующим
вычислительной сложности всей сети. Зарекомендовавшее себя средство, оно не
является выгодным с точки зрения расхода природных ресурсов, поскольку объём
энергозатрат для выполнения вычислений, необходимых на добавление одного блока,
сопоставим с объемом энергии, потребляемым двумя средними американскими домами (Х).
\subsubsection{Proof of Stake (PoS)}
Алгоритм доказательства доли владения --- основной конкурент предыдущего
алгоритма. Вероятность формирования участником очередного блока в реестре
пропорциональна доле, которую составляют принадлежащие этому участнику
расчётные единицы данного реестра (напр., кол-во единиц криптовалюты) от их
общего количества. Поскольку для принятия решения о том, кто будет являться
создателем нового блока, нет необходимости в большом объёме вычислений, как в
PoW, зачастую является выбором, когда экологический фактор поставлен на одно из
главных мест.
\subsubsection{Delegated Proof-of-Stake (DPoS)}
\subsubsection{Proof-of-Authority (PoA)}
\subsubsection{Proof-of-Weight (PoWeight)}
\subsubsection{Byzantine Fault Tolerance (BFT)}
\subsubsection{Consensus in DAGs}


\newpage
\section{Глава 2. Проектирование сервиса: архитектура, процессы}
В данной главе рассматриваются особенности реализации сервиса для
автоматизации программирования с использованием технологии блокчейн на основе
различных алгоритмов. В начале приведены архитектурные составляющие проекта
\textbf{с обоснованием использования} средств реализации приведённой
архитектуры, далее --- функциональные, а в заключении собраны выводы по главе.

\subsection{Архитектурные особенности}
\subsubsection{Общая структура проектного решения}
Глобально проект состоит из двух приложений. Первый отвечает за интерактивное
взаимодействие с пользователем и генерацию кода второго с
использованием указанных пользователем алгоритмов (далее --- компоновщик).
Вторая --- это имплементация блокчейна (далее --- реализация блокчейна).
Приложение содержит код для кошелька ({\small wallet.py}) и майнера ({\small
miner.py}) с определённой функциональностью. Код второго приложения
структурирован для удовлетворения нужд использования указанных пользователем
методов. Методы и классы генерируются at-runtime первого приложения.

\paragraph{Язык программирования}
Использованный язык программирования --- \textbf{Python} верисии
\underline{3.6.5}.\\
\emph{Рассматриваемые аналоги}: C, Java\\\\
Выбран вследствие своей универсальности применения
относительно (а) алгоритмов, (б) платформы для запуска; а так же простоты
реализации побочных, вспомогательных компонент, посредственно относящихся к
данному проекту. Реализация их на других языках была бы необходимостью --- и, в
следствии их посредственного отношения к проекту, необоснованной тратой
временного ресурса.\\
\emph{Реализации алгоритмов \textbf{keccak-256 \emph{ и } keccak-512} были
переписаны с Python2 на Python3.6.5 для совместимости с данной программой}.

\paragraph{Тип приложения}
Приложение является консольной утилитой, которая может быть установлена в
систему семейства \underline{GNU/Linux} при помощи программы
\textbf{python3-pip}.\\
\emph{Рассматриваемые аналоги}: Приложение с графическим интерфейсом,
приложение в веб-интерфейсом\\\\
Отсутствие графического интерфейса обосновано отсутствием
необходимости произведения манипуляций при помощи мыши, и отсутствием
необходимости отображения графических изображений и другой информации.
Интерактивный диалог производится посредством вывода в стандартный выход
(\emph{stdout}) консоли текста с опциями; а выбор пользователя регистрируется
посредством считывания стандартного ввода (\emph{stdin}).

\paragraph{Протокол обмена данными между компонентами}
Выбранной опцией является передача \textbf{json} файлов посредством \textbf{http} протокола.\\
\emph{Рассматриваемые аналоги}: \textbf{grpc}
(http://flask.pocoo.org/docs/0.12/api/\#flask.Flask), \textbf{zmq}
(http://zeromq.org/)\\\\
Обеспечивается использованием модуля \emph{Flask} и \emph{json}. В сравнении
с \emph{grpc} (https://grpc.io/) и \textit{zmq} протоколами, \emph{http}
представлялся наиболее подходящим вследствие своей популярности и удобства
подключения и использования совместно с языком Python.

\paragraph{Хранилище}
В качестве хранилища было реализовано \textbf{key-value} (ключ-значение) хранилище.\\
\emph{Рассматриваемые аналоги}: \textbf{etcd} (https://coreos.com/etcd/), \textbf{sqlite} (https://sqlite.org/index.html)\\\\
Подключение сторонней библиотеки или базы данных сильно увеличило бы вес
приложения в целом, а так же добавило бы ещё несколько зависимостей. Вследствие
этого было решено придумать импровизированное key-value хранилище на основе
структуры данных словарь (dictionary).
Реализация и функциональность описана в настоящем техническом задании
(Приложение REF).

\paragraph{Автообновление}
Задачей автообновления занимается UNIX утилита \textbf{cron}.\\
\emph{Рассматриваемые аналоги}: Импровизированный планировщик событий\\\\
В связи с существованием качественного и надёжного решения в лице
\textbf{cron}'a, было решено использовать его. Настроен на сервере. Подробнее
--- в п. \ref{autoonova}

\paragraph{Continuous integration}
В качестве средства CI был выбран \textbf{Shippable} (app.shippable.com).\\
\emph{Рассматриваемые аналоги}: \textbf{TravisCI} (https://travis-ci.org/), \textbf{CircleCI} (https://circleci.com/)\\\\
Данное средство распространяется с возможностью использовать бесплатную версию
программы, поэтому выбор пал на неё. \emph{Необходимость} в сервисе CI
выражается в том, чтобы данное приложение генерировало рабочие стабильные коды
даже после обновления используемых реализаций алгоритмов, что не зависит от
разработчика.

\paragraph{Конфигурация}
Конфигурационный файл программы написан на языке \textbf{YAML}.\\
\emph{Рассматриваемые аналоги}: \textbf{JSON}, \textbf{Plain text}\\\\
Выбран \emph{YAML} по причине хорошей поддерживаемости в Python и своей эстетической
непритязательности по сравнению с аналогами.




\subsubsection{Архитектура компоновщика}
Компоновщик --- часть проекта, автоматизирующая процесс программирования и
позволяющяя тем самым создавать готовые решения. Решением может быть рабочий
код блокчейна с использованием 24 вариаций алгоритмов.

\subsubsection{Порядок работы компоновщика}
Программно компоновщик был назван {\small gsl}. Основная команда с которой
придётся иметь дело --- init. Пример:\\

\begin{center}
\begin{Verbatim}[frame=single]
gsl --init --name myledger --path ~/tmp/gsl
\end{Verbatim}
\end{center}

\begin{figure}[h]
    \centering
    \includegraphics[width=\textwidth]{images/sequence}
    \caption{Sequence диаграмма последовательности работы первой компоненты --- компоновщика}\label{sequence}
\end{figure}

По вызову этой команды происходят процессы, обозначенные на диаграмме
последовательностей (Рис. \ref{sequence}). Зачитывается, загружается в
программу и валидируется конфигурационный файл программы.

\newpage

Затем начинается диалог с пользователем (Рис. \ref{dialog}).
\begin{figure}[h]
    \centering
    \includegraphics[width=\textwidth]{images/dialog_start}
    \caption{Начало диалога выбора алгоритмов}\label{dialog}
\end{figure}

В этом диалоге пользователь выбирает какие алгоритмы хэширования и цифровой
подписи будут использованы в его будущей реализации блокчейна. После этого,
пользователю предоставляется выбор остальных параметров блокчейна, по которым ему
будут предложены справочные ссылки для изучения (Рис. \ref{sprav}).  После
выбора происходит установка данных библиотек и генерация кода реализации
блокчейна по указанному пути (Рис. \ref{ll}).

\begin{figure}[h]
    \centering
    \includegraphics[width=\textwidth]{images/spravochno}
    \caption{Справочная информация по выбранным параметрам}\label{sprav}
\end{figure}

\subsubsection{Архитектура реализации блокчейна}
\begin{figure}[h]
    \centering
    \includegraphics[width=0.8\textwidth]{images/ledger_ll}
    \caption{Директория со сгенерированным кодом реализации блокчейна}\label{ll}
\end{figure}

Код реализации блокчейна запускается интерпретатором языка Python 3.6.5.
Скрипты {\small miner.py} и {\small wallet.py} запускаются без аргументов
командной строки. Запустив {\small miner.py} (Рис. \ref{miner_run}), можно запускать {\small
wallet.py} (Рис. \ref{wallet_run}), в котором есть возможности:
\begin{enumerate}
    \item Сгенерировать кошелёк: пару публичный-приватный ключи и записать их в файл
    \item Отправить с одного кошелька на другой N условных едениц
    \item Провалидировать транзакции
\end{enumerate}

\begin{figure}[h]
    \centering
    \includegraphics[width=\textwidth]{images/miner_run}
    \caption{Лог запуска майнера}\label{miner_run}
\end{figure}

\begin{figure}[h]
    \centering
    \includegraphics[width=0.8\textwidth]{images/wallet_run}
    \caption{Возможности кошелька}\label{wallet_run}
\end{figure}

Генерация пары ключей, а также хэширование записей происходит посредством
использования выбранных ранее пользователем алгоритмов.

\newpage
\subsubsection{Работа с данными}\label{dannie_sheme}
Исходный код алгоритмов хранится в директории {\small src/altorithms/hashing} и
{\small src/altorithms/digital\_signature}. Он собирается полным проходом в
сеть по ссылкам, расположенными в импровизированном key-value хранилище
(описано в п. \ref{shron}, а также в настоящем техническом задании ---
Приложение 1).
% Приложение \ref{tz}).
Данная процедура происходит при автообновлении алгоритмов
на сервере каждый день в 21:00 (Рис. \ref{update}). После процедуры
автообновления, пользователи могут по желанию обновить свою версию программы и
использовать более свежий код.

\begin{figure}[h]
    \centering
    \includegraphics[width=\textwidth]{images/server}
    \caption{Процесс работы сервера автообновления}\label{update}
\end{figure}

\subsection{Функциональные особенности}


\newpage
\section{Глава 3. Программная реализация}
\subsection{Функциональные требования}
К функциональным требованиям приложения компоновщик относятся:

\begin{enumerate}
    \item Вывод информации в цвете, обозночающий степень поддержки программой алгоритма
    \item Генерирование значений для выбора ``по умолчанию''
    \item Возможность записать выбор пользователя
    \item Возможность поиска в хранилище ссылок для конкретных алгоритмов
    \item Возможность загрузки из общедоступных источников исходных кодов алгоритмов
    \item Возможность установки загруженных алгоритмов на ФС машины без особых прав
    \item Возможность генерировать код по указанной директории
    \item Возможность замера времени работы выбранных алгоритмов
    \item Возможность просмотра справочной информации по остальным параметрам реестра
\end{enumerate}

К функциональным требованиям приложения реализация блокчейна ({\small wallet.py})относятся:
\begin{enumerate}
    \item Возможность генерации ``кошелька'' --- пары приватный + публичный ключ
    \item Использование в качестве алгоритма цифровой подписи выбранный пользователем
    \item Использование в качестве алгоритма хэширования выбранный пользователем
    \item Возможность записи данных ``кошелька'' на ФС машины
    \item Возможность отправки от одного пользователя другому ограниченного
          количества условной криптовалюты
    \item Возможность проверки цепочки транзакций на валидность
    \item Возможность записи времени исполнения своих операций в файл при
          наличии параметра \_profd и \_timed
\end{enumerate}

К функциональным требованиям приложения реализация блокчейна ({\small wallet.py})относятся:
\begin{enumerate}
    \item Возможность принимать json сообщения по http протоколу
    \item Использование в качестве алгоритма цифровой подписи выбранный пользователем
    \item Использование в качестве алгоритма хэширования выбранный пользователем
    \item Возможность выполнения proof-of-work алгоритма
    \item Возможность вычисления подходящего хэш значения (майна) блока
    \item Возможность добавления блока в цепочку
    \item Возможность записи времени исполнения своих операций в файл при
          наличии параметра \_profd и \_timed
\end{enumerate}


\subsection{Описание реализации процесса генерации кода компоновщиком}\label{komponovshik}
Процесс состоит из нескольких частей:
\begin{enumerate}
    \item Сбор опций пользователя при помощи интерактивного диалога
    \item Поиск опций в хранилище
    \item По путям из хранилища программа осуществляет поход в интернет и
          загрузку кодов алгоритмов
    \item Загруженные алгоритмы записываются на ФС машины
    \item Загруженные алгоритмы устанавливаются в систему
    \item На ФС записываются функции, обеспечивающие совместимость выбранных
          пользователем алгоритмов и реализации блокчейна
    \item На ФС записывается реализация блокчейна
\end{enumerate}

Процессы будут описываться по порядку, начиная со сбора опций.

\subsubsection{Сбор опций пользователя}

Начинается с отображения приветственного и инструкционного сообщения (Рис. \ref{dg_st})

\begin{figure}[h]
    \centering
    \includegraphics[width=\textwidth]{images/dialog_start}
    \caption{Начало работы компоновщика: приветственное окно и первый набор алгоритмов}\label{dg_st}
\end{figure}

В листинге \ref{print_opts} описан процесс сбора опций пользователя. Из
хранилища берётся список всех возможных опций, и согласно порядку типов
алгоритмов, распечатываются в пронумерованном виде соответствующие опции.
Пользователь затем вводит номер, размах значений которого соответствует опциям
алгоритмов данного типа. Предусмотрено значение ``по умолчанию'' для каждой из
опций --- первое значение. Вывод алгоритмов раскрашивается --- из
реализованного в рамках данной работы \textbf{utils} были импортированы функции
и константы для форматирования вывода. В случае ошибки в стандартный выход
логгера выводится её сообщение.
\begin{center}
\begin{lstlisting}
    for k, v in OPTIONS.items():
        print(f'\nChoose type of {k} of the ledger')
        if isinstance(v, list):
            for num, opt in enumerate(v):
                if 'hash' in k or 'digital' in k:
                    if opt in TOINSTALL:
                        prefix = ASCIIColors.BACK_BLUE
                    else:
                        prefix = ASCIIColors.BACK_LIGHT_BLUE
                else:
                    prefix = ASCIIColors.ENDS
                print(f'{prefix}{num+1}: {opt}{ASCIIColors.ENDS}', end='\n')
            try:
                n = input(f'Enter num from 1 to {len(v)}, default [1]: ')
                n = 0 if n == '' else int(n) - 1
                if n < 0 or n >= len(v):
                    raise ChooseError
                self.ledger_config[k] = n
            except Exception as e:
                __logger__.exception(str(e))
                return
        else:
            print(v)
\end{lstlisting}\label{print_opts}
    Листинг \ref{print_opts}: Процесс выбора опций пользователем
\end{center}

\subsubsection{Поиск алгоритмов}
На данный момент выбранные пользователем опции были получены, и записаны в переменную \textbf{options}.
Поиском алгоритмов в хранилище, интернете и их установкой на ФС машины занимается класс \textbf{ProlificWriter}.

Процесс поиска ссылок для алгоритмов в хранилище происходит путём поиска
соответствий переданных классу опций с полем хранилища \emph{TOINSTALL}. Запись
на ФС установленных алгоритмов происходит в методе \textbf{write}. Сначала
пишутся алгоритмы хэширования и электронной подписи, а затем --- код реализации
блокчейна.

\begin{center}
\begin{lstlisting}
    def write(self):
        # WRITE ALGORITHMS ITSELF
        #
        self._write_hashing_()
        self._write_digital_signature_()

        self._write_(wallet)
        self._write_(miner)


    def _write_(self, script_to_write):
        name = f'{script_to_write.__name__.split(".")[-1]}.py'
        src_code = getsource(script_to_write)
        with open(os.path.join(self.path, name), 'w') as __fd__:
            __fd__.write(src_code)
        if self.timed:
            os.system('sed -ir "0,/def _timed/{s/_timed = .*/_timed = True/}" ' + os.path.join(self.path, name))
        if self.profd:
            os.system('sed -ir "0,/def _profd/{s/_profd = .*/_profd = True/}" ' + os.path.join(self.path, name))


    def _write_hashing_(self):
        # INSTALLING PROCEDURE
        src_path = self._get_src_path_()
        path = os.path.join(src_path, get('TOINSTALL', self.opts['hashing']))
        self._install_(path)

        # WRITING PROCEDURE
        name = 'myhashing.py'
        type_ = self.opts['hashing']
        path = os.path.join(src_path, get('INTERFACES', type_), name)
        if not os.path.exists(self.path):
            os.makedirs(self.path)
        copyfile(path, os.path.join(self.path, name))

    def _write_digital_signature_(self):
        # INSTALLING PROCEDURE
        src_path = self._get_src_path_()
        path = os.path.join(src_path, get('TOINSTALL', self.opts['digital signature']))
        self._install_(path)

        # WRITING PROCEDURE
        name = 'mydss.py'
        type_ = self.opts['digital signature']
        path = os.path.join(src_path, get('INTERFACES', type_), name)
        copyfile(path, os.path.join(self.path, name))
\end{lstlisting}\label{write}
    Листинг \ref{write}: Процесс поиска алгоритмов и записи на ФС машины
\end{center}

\subsubsection{Установка}
Установка алгоритмов, упомянутая до этого в листинге \ref{write}, происходит
посредством запуска метода \_install\_, а при неудачной попытки ---
\_pip\_install\_.

\begin{center}
\begin{lstlisting}
    def _install_(self, path):
        """
        Installs with `python setup.py install`
        """
        os.chdir(path)
        subprocess.call([sys.executable, f'{path}/setup.py', 'install'])


    def _pip_install_(self, package):
        subprocess.call([sys.executable, '-m', 'pip', 'install', package, '--user'])
\end{lstlisting}\label{lst:inst}
Листинг \ref{lst:inst}: Код для установки алгоритмов на ФС машины
\end{center}


\subsubsection{Запись реализации блокчейна}

Запись на ФС методов-обёрток для совместимости выбранных пользователем
алгоритмов и реализации блокчейна, содержится в листинге \ref{write}. За запись
методов-обёрток отвечает метод \textbf{\_write\_hashing\_} и
\textbf{\_write\_digital\_signature\_}.




\subsection{Описание реализации блокчейна}
Реализация блокчейна делится на 2 модуля: {\small wallet.py} и {\small
miner.py}. Описание каждого из них представлено в данной секции.
Данные модули генерируются компоновщиком, описанным в п. \ref{komponovshik}.
Компоновщик сгенерировал данные модули таким образом, что они используют
реализации алгоритмов, выбранных пользователем.

\subsubsection{wallet.py}
Модуль, позволяющий выполнить по запуску со стороны пользователя одну из трёх
функциональностей:

\begin{enumerate}
    \item Создать новый кошелёк
    \item Отправить средств
    \item Валидировать транзакции
\end{enumerate}


Модуль должен использовать алгоритмы, выбранные пользователем на этапе
генерации. Поэтому в начале, алгоритм импортирует их (листинг
\ref{lst:import}). В противном случае, при невозможности их импортирования,
модулем будут применены стандартные алгоритмы.  Подобными для хэширования
является \textbf{SHA-256}, а для цифровой подписи \textbf{ECSDA}. 

\begin{center}
\begin{lstlisting}
try:
    import mydss
    dss = mydss
    if hasattr(dss, 'name') and hasattr(dss, 'bit'):
        alg_name = dss.name
        alg_bit = dss.bit
        try:
            from mydss import mydss
            dss = mydss
        except:
            dss = mydss
except:
    import ecdsa
    dss = ecdsa
    alg_name = 'ecdsa'
    alg_bit = '256'
\end{lstlisting}\label{lst:import}
    Листинг \ref{lst:import}: Импортирование выбранных пользователем опций (myhashing и mydss)
\end{center}

\paragraph{Создание нового кошелька}
При создании нового кошелька, запрашивается имя пользователя, и генерируются
публичный и приватный ключи. Далее, они записываются в файл
\textbf{<name>.txt}.  Генерация ключей осуществляется в методе {\small
generate\_ECDSA\_keys(ret=False)}.  При генерации по умолчанию, используется
алгоритм \textbf{ECSDA}. При успешной попытке использования указанного
пользователем алгоритма, используется он. Для компактности записи, длинный
публичный ключ кодируется в base64 и в таком формате записыватеся в файл. В
дальнейшем, он будет декодирован для корректной процедуры отправки/верификации.
Описанная процедура представлена в листинге \ref{lst:creation}.

\begin{center}
\begin{lstlisting}
def generate_keys(ret=False):
    if _timed:
        t1 = time.time()
    singningkey = dss.SigningKey().generate()
    private_key = singningkey.to_string()
    vk = singningkey.getverifyingkey()
    public_key = vk.to_string().hex()
    public_key = singningkey.to_string(pub=True)
    if _timed:
        t2 = time.time()
        _write_time(alg_name, 'Key pair generation', alg_bit, t2-t1)
    public_key = base64.b64encode(bytes.fromhex(public_key))
    if ret:
        return private_key, public_key.decode()
    filename = input('Name of addr: ') + '.txt'
    with open(filename, 'w') as f:
        f.write('PrvK: {0}\nWallet addr / PubK: {1}'.format(private_key, public_key.decode()))
    print('saved{0}'.format(filename))
\end{lstlisting}\label{lst:creation}
    Листинг \ref{lst:creation}: Генерация пары публичного и приватного ключей
\end{center}



\paragraph{Отправка средств}
При отправки средств, используются стандартные подходы блокчейна. В модуле
{\small miner.py} происходит (Листинг \ref{lst:send}) основная обработка и
добавление самого блока к общей цепочке. В рассматриваемом {\small wallet.py},
происходит лишь сбор всей необходимой для совершения транзакции в json payload
и отправляется на URL в API модуля {\small miner.py}. Дальнейшая обработка
отправленного запроса происходит в \ref{lst:newblock}.

\begin{center}
\begin{lstlisting}
def _perform_transaction(from_, prv_key, addr_to, amount):
    len_prv = len(prv_key)

    if dss.name == 'gost' or len_prv == 64:
        signature, message = _sign_msg(prv_key)
        url = f'http://localhost:{_port}/mycoin'
        payload = {'from': from_,
                   'to': addr_to,
                   'amount': amount,
                   'signature': signature.decode(),
                   'message': message}
        headers = {'Content-Type': 'application/json'}

        res = requests.post(url, json=payload, headers=headers)
        print(res.text)
    else:
        print('Wrong address; Please try again.')
\end{lstlisting}\label{lst:send}
    Листинг \ref{lst:send}: Отправка сформированного блока в {\small miner.py}
\end{center}


\paragraph{Валидация трназакций}
Валидация транзакций в скрипте {\small wallet.py} ограничивается отправкой в
{\small miner.py} запроса на проверку блоков (Листинг \ref{lst:valid}).

\begin{center}
\begin{lstlisting}
    def check_transactions():
        res = requests.get(f'http://localhost:{_port}/blocks')
        print(res.text)
\end{lstlisting}\label{lst:valid}
    Листинг \ref{lst:valid}: Отправка запроса в {\small miner.py} на проверку
    валидности транзакций
\end{center}


\subsubsection{miner.py}
Данный модуль запускается и работает как фоновый процесс во время отправки и
валидации сообщений первого. Он выполняет такие операции как mine block (майннг
блока) --- процесс, в котором происходит вызов proof-of-work процедуры, после
которой считается возможным добавить в цепочку блок.

Со стороны http api для {\small wallet.py}, он выполняет:
\begin{enumerate}
    \item Принимает \textbf{POST} запрос на добавление нового блока в цепочку
    \item Принимает \textbf{GET} запрос на проверку существующих блоков в
          цепочке
\end{enumerate}

\paragraph{Создание нового блока}
Создание нового блока происходит в несколько этапов.

Информация о входящей транзакции регистрируется в очереди {\small
NODE\_PENDING\_TRANSACTIONS} и выводится на стандартный вывод (stdout) ---
Листинг \ref{lst:newblock}.  Процесс, запущенный в отдельном потоке исполнения
ОС, целевой функцией которого является mine, получает изменения из очереди
{\small NODE\_PENDING\_TRANSACTIONS} и запускает процесс майна нового блока ---
Листинг \ref{lst:mine}.

\begin{center}
\begin{lstlisting}
    @app.route('/mycoin', methods=['GET', 'POST'])
    def transaction():
        """Each transaction sent to this node gets validated and submitted.
        Then it waits to be added to the blockchain. Transactions only move
        coins, they don't create it.
        """
        if request.method == 'POST':
            new_mycoin = request.get_json()
            if validate_signature(new_mycoin['from'], new_mycoin['signature'], new_mycoin['message']):
                WAITING_TRANSACTIONS.append(new_mycoin)
                print("Got a new transaction")
                print("FROM: {0}".format(new_mycoin['from']))
                print("TO: {0}".format(new_mycoin['to']))
                print("AMOUNT: {0}\n".format(new_mycoin['amount']))
                return "Transaction submission successful\n"
            else:
                return "Transaction submission failed. Wrong signature\n"
        elif request.method == 'GET' and request.args.get("update") == MINER_ADDRESS:
            pending = json.dumps(WAITING_TRANSACTIONS)
            WAITING_TRANSACTIONS[:] = []
            return pending
\end{lstlisting}\label{lst:newblock}
    Листинг \ref{lst:newblock}: URL '/txion' в {\small miner.py} для создания нового блока
\end{center}


\begin{center}
\begin{lstlisting}
def mine(a, blockchain, WAITING_TRANSACTIONS):
    BLOCKCHAIN = blockchain
    WAITING_TRANSACTIONS = WAITING_TRANSACTIONS
    while True:
        if _timed:
            t1 = time.time()
        last_block = BLOCKCHAIN[len(BLOCKCHAIN) - 1]
        last_proof = last_block.data['proof-of-work']
        proof = proof_of_work(last_proof, BLOCKCHAIN)
        if not proof[0]:
            BLOCKCHAIN = proof[1]
            a.send(BLOCKCHAIN)
            continue
        else:
            WAITING_TRANSACTIONS = requests.get(MINER_NODE_URL + "/txion?update=" + MINER_ADDRESS).content
            WAITING_TRANSACTIONS = json.loads(WAITING_TRANSACTIONS)
            WAITING_TRANSACTIONS.append({
                "from": "network",
                "to": MINER_ADDRESS,
                "amount": 1})
            new_block_data = {
                "proof-of-work": proof[0],
                "transactions": list(WAITING_TRANSACTIONS)
            }
            WAITING_TRANSACTIONS = []
            new_block_index = last_block.index + 1
            new_block_timestamp = time.time()
            last_block_hash = last_block.hash
            mined_block = Block(new_block_index, new_block_timestamp, new_block_data, last_block_hash)
            BLOCKCHAIN.append(mined_block)
            try:
                print(json.dumps({
                  "index": new_block_index,
                  "timestamp": str(new_block_timestamp),
                  "data": new_block_data,
                  "hash": last_block_hash.decode()
                }) + "\n")
            except:
                print(json.dumps({
                  "index": new_block_index,
                  "timestamp": str(new_block_timestamp),
                  "data": new_block_data,
                  "hash": last_block_hash
                }) + "\n")
            a.send(BLOCKCHAIN)
            requests.get(MINER_NODE_URL + "/blocks?update=" + MINER_ADDRESS)
            if _timed:
                t2 = time.time()
                _write_time(hash_name, 'Mining one block', hash_bit, t2-t1)
\end{lstlisting}\label{lst:mine}
    Листинг \ref{lst:mine}: Процесс майнинга новых блоков
\end{center}


\paragraph{Валидация подписи}
Валидация электронной подписи происходит в методе {\small validate\_signature}.
В нём вызываются методы алгоритма цифровой подписи. Алгоритм может быть либо
выбранный пользователем на этапе генерации, или, если его не происходило,
стандартным, то есть \textbf{ECSDA} с кривой  SECP256k1 --- Листинг
\ref{lst:validate}.


\begin{center}
\begin{lstlisting}
def validate_signature(public_key, signature, message):
    if _timed:
        t1 = time.time()
    public_key = (base64.b64decode(public_key)).hex()
    signature = base64.b64decode(signature)
    vk = dss.VerifyingKey().from_string(public_key)
    try:
        res = vk.verify(signature, message.encode())
        if _timed:
            t2 = time.time()
            _write_time(alg_name, 'Verifying signature', alg_bit, t2-t1)
        return res
    except:
        return False
\end{lstlisting}\label{lst:validate}
    Листинг \ref{lst:validate}: Процесс верификации электронной подписи
\end{center}


\subsection{Описание процесса автообновления}
Принципиальная схема работы автообновления представлена в п. \ref{dannie_sheme}.\\
Автообновление настроено на отдельном сервере по расписанию. Конфигурация
сервера представляет из себя VPS виртуальную машину, на которой установлена OS
\emph{Ubuntu 16.04 LTS} с версией ядра \emph{4.4.0-148-generic}.

Расписание автообновления сконфигурировано при помощи стандартной UNIX утилиты
\textbf{cron}.  Конфигурация \textbf{crontab} представлена в листинге
\ref{lst:cron}. Происходят там следующие процессы:

\begin{itemize}
    \item Каждый день в 20:00 запускается скрипт {\small updater.py}, который получает актуальные изменения алгоритмов в локальную копию на серверную ФС;
    \item Каждый день в 21:00 запускается скрипт {\small pusher.py}, которые полученные изменения загружает в публичный репозиторий, делая обновлённые и актуальные данные доступными всем;
    \item Логи обновления и загрузки пишутся в {\small updater.log} и {\small pusher.log} соответственно.
\end{itemize}

\begin{center}
\begin{lstlisting}
        00 20 * * * echo $(date) >> /home/coldmind/gsl/updater.log && cd /home/coldmind/gsl/src && ./updater.py >> /home/coldmind/gsl/updater.log 2>&1
        00 21 * * * echo $(date) >> /home/coldmind/gsl/pusher.log && cd /home/coldmind/gsl && ./src/pusher.sh >> /home/coldmind/gsl/pusher.log 2>&1
\end{lstlisting}\label{lst:cron}
    Листинг \ref{lst:cron}: Конфигурация {\small crontab -l}
\end{center}

Код скрипта обновления {\small updater.py} представлен в листинге
\ref{lst:updater}. Он использует реализованное key-value хранилище (\ref{dannie_sheme}) для
получения ссылок, по которым возможно обновление загруженных алгоритмов, и для
каждой ссылки и пути записи в ФС (тоже получает из key-value хранилища),
запускает скрипт {\small pull\_single.sh} (Листинг \ref{lst:pull_single}).

\paragraph{updater}

\begin{center}
\begin{lstlisting}
#!/usr/bin/env python3.6

import subprocess
from technologies import _kv_, get
toinstall = _kv_.toinstall
update_links = _kv_.update_links

for alg, src in update_links.items():
    p = subprocess.popen(['bash', 'pull_single.sh', f'{get("toinstall", alg)}', f'{src}'], stdout=subprocess.pipe)
    (result, error) = p.communicate()
    print(result.decode())
\end{lstlisting}\label{lst:updater}
    листинг \ref{lst:updater}: Скрипт обновления всех алгоритмов со внешних источников
\end{center}

Скрипт {\small pull\_single.sh} (Листинг \ref{lst:pull_single}) представляет из
себя поход по ссылке во внешний источник, выгрузку данных оттуда, и замещение
новыми данными старые.

\begin{center}
\begin{lstlisting}
#!/usr/bin/env bash

if cd $1;
then
    cd ..
    name1=$(echo $1 | rev | cut -d/ -f1 | rev)
    name2=$(echo $2 | rev | cut -d/ -f1 | rev)
    if [[ $name1 == $name2 ]]
    then
        rm -rf $name1
        git clone $2
        cd $name2
        rm -rf .git*
    fi
fi
\end{lstlisting}\label{lst:pull_single}
    листинг \ref{lst:pull_single}: Скрипт обновления единичного алгоритма
\end{center}


\paragraph{pusher}

Скрипт {\small pusher.py} используется для загрузки полученных изменений в сеть
для общего доступа. Представлен в листинге \ref{lst:pusher}. При неизвестной
ошибке в процессе обновления или некорректной работе обновлённых алгоритмов,
подключенный к репозиторию сервис Continuous Integration не позволит вступить
изменениям в силу, вследствие чего, пользователи будут защищены от нерабочего
кода.

\begin{center}
\begin{lstlisting}
#!/usr/bin/env bash

curr_date=$(date +%d_%m_%Y)

git add .
git commit -m "Update algorithms: $curr_date."
git push


# EOF
\end{lstlisting}\label{lst:pusher}
    листинг \ref{lst:pusher}: Скрипт для загрузки обновлённых данных
\end{center}
\label{autoobnova}

\subsection{Описание реализации хранилища данных}\label{shron}
Хранилище представляет собой импровизированное key-value хранилище, выбор среди
альтернатив которого, описан в п. \ref{hraaan}.

Данные, которыми располагает приложение, подразумевают внесение изменений лишь
разработчиком приложения. Этими данными являются ссылки на сетевые адреса
хранения исходных кодов алгоритмов, пути записи алгоритмов в ФС компьютера, и
пути до функций-обёрток данных алгоритмов. Все представленные данные, при
необходимости, правятся разработчиком, в следствие чего, хранилище имеет только
метод \textbf{get} (Листинг \ref{lst:get}), и является read-only. Entity-relationship диаграмма
представлена на рисунке \ref{er}.

\begin{figure}[h]
    \centering
    \includegraphics[width=\textwidth]{images/er}
    \caption{Диаграмма отношений сущностей в реализованном хранилище}\label{er}
\end{figure}

\newpage
Метод \textbf{get} хранилища:

\begin{center}
\begin{lstlisting}
def get(name, val, default=None):
    return getattr(_kv_, name).get(val, default)
\end{lstlisting}\label{lst:get}
Листинг \ref{lst:get}: код метода get хранилища
\end{center}

Код класса хранилища:
\begin{center}
\begin{lstlisting}
class _kv_(object):
    OPTIONS = {
            'hashing':  ['SHA-256', 'SHA-512', 'Scrypt', 'KECCAK-256',
                         'KECCAK-512', 'Ethash', 'X11', 'X17', 'myr-groestl',
                         'Lyra2rev2', 'blake2s', 'blake2b'],
#...                    
    }

    LINKS = { } # ....

    UPDATE_LINKS = { } # ....

    TOINSTALL = { } # ....

    INTERFACES = { } # ....
    \end{lstlisting}\label{lst:hran}
    Листинг \ref{lst:hran}: Представление кода реализованного key-value хранилища
\end{center}





\newpage
\section{Глава 4. Проведение эксперементов}
\subsection{Анализ времени исполнения алгоритмов в реализации блокчейнов}
Была предпринята попытка замерить время исполнения всех 24 вариаций алгоритмов на следующих операциях:

\begin{itemize}
\item Генерация пары ключей (публичный-приватный)
\item Вычисления хэш-значения
\item Электронная подпись сообщения
\item Верификация сообщения
\item Время работы Proof of Work
\item Время майна одного блока
\end{itemize}

Были замерены времена исполнений участков кода готовых реализаций блокчейнв,
сгенерированных при помощи разработанного компоновщика.
Замеры проводились при помощи модуля \emph{time} в Python. Дальнейший анализ и
построение графиков происходило в среде \emph{Jupyter Notebook} с
использованием модуля \underline{matplotlib}. Подробнее о методе замера в главе 3 (REF).

Была составлена таблица с записями вида \emph{[название алгоритма; функция; битрейт;
время исполнения]} размером более 1500 записей.

\begin{figure}
    \centering
    \includegraphics[width=\textwidth]{./images/boxes}
    \caption{Box-plot'ы для распределения алгоритмов по времени}\label{boxes}
\end{figure}

Далее были построены распределения общего вида (Рис. \ref{boxes}), а так же
гистограммы распределения времени выполнения для алгоритмов цифровой подписи,
разделяя их на вышеперечисленные процессы (Рис. \ref{dss}):

\begin{figure}[h]
    \centering
    \includegraphics[width=0.68\textwidth]{./images/hists_dss}
    \caption{Распределение времени выполнения среди алгоритмов цифровой подписи}\label{dss}
\end{figure}

И, аналогичные данные быи построены для алгоритмов хэширования (Рис. \ref{hash}):

\begin{figure}
    \centering
    \includegraphics[width=\textwidth]{./images/hists_hashing}
    \caption{Распределение времени выполнения среди алгоритмов хэширования}\label{hash}
\end{figure}

С целью сравнить время исполнения алгоритмов по одинаковым процессам,
расположим их на одном графике гистограммы.  На Рис. \ref{boxes} с бокс-плотами
были выдны выбрасы, поэтому для усреднения значений на последующих графиках,
были взяты медианы значений, поскольку данные показатель более устойчив к
выбросам, чем обычное среднее.

\begin{figure}[h]
    \centering
    \includegraphics[width=\textwidth]{./images/hash_comparison}
    \caption{Сравнение времени исполнения работ различных алгоритмов хэширования на одинаковых функциях}\label{hash_comp}
\end{figure}

Самым быстрым алгоритмом хэширования \emph{среди упомянутых}, исходя из
проведённого анализа, является, SHA-256. Самым медленным из реализованных на
языке СИ --- X17.  На данных графиках интересно заметить, что оба алгоритма,
реализованные на языке Python, имеют гораздо большее время исполнения, по
сравнению с реализованными на языке СИ. Алгоритм ГОСТ 34.10-2012 проигрывает
международно известной ECSDA лишь во времени исполнения процедуры
верификации (Рис. \ref{dss_comp}).

\begin{figure}[h]
    \centering
    \includegraphics[width=1.\textwidth]{./images/dss_comparison}
    \caption{Сравнение времени исполнения работ различных алгоритмов цифровой подписи на одинаковых функциях}\label{dss_comp}
\end{figure}


\newpage
\subsection{Выводы}
Рассмотренные временные показатели алгоритмов могут дать пользователю
представление о том, каким будут временные показатели его блокчейна. Данное
сравнение позволит пользователю ориентироваться при выборе желаемой
конфигурации.


\newpage
\section{Заключение}
% В работе были рассмотрены использования современных алгоритмов и протоколов в
% распределенных реестрах, была актуализирована их классификация. Наличие такого
% объёма алгоритмов по защите конфиденциальности и приватности сигнализирует о
% явных проблемах в данной области. И действительно, как показало исследование,
% даже такие алгоритмы как Coinshuffle и Stealth Address не гарантируют
% анонимность отправителей и получателей. Реализация современных распределённых
% реестров предполагает гибкость и адаптируемость под новые алгоритмы и
% протоколы, которые появляются каждый месяц. В будущем мы надеемся увидеть
% прогресс в области разработки новых (основанных на свежих идеях) алоритмов и
% протоколов, что приведёт к скачку и новому всплеску популярности распределённых
% реестров в обществе.

% Написанная библиотека доступна для свободной установки и помогает разобраться в
% практических реализациях написанных алгоритмов, а так же даёт возможность
% построить учебную версию распределённого реестра. %(\ref{pril1}).


Было проведено исследование использования различных алгоритмов и протоколов в
распределённых реестрах. Были найдены новые типы распределённых реестров, такие
как Holochain, Hashgraph и Tempo. Это позволило решить проблему устаревшей
информации по классификации алгоритмов и протоколов в распределённых реестрах.\\
Была разработана актуальная классификация использования алгоритмов и протоколов
в распределённых реестров, в которой отражено современное многообразие
технологичных подходов к решению проблем безопасности.

Было разработано гибкое, масштабируемое программное средство для автоматизации
работы программирования. Разработан уникальный процесс по работе с исходными
кодами алгоритмамов, расположенных удалённо, их использованию и автообновлению.
Налажена самоподдерживаемая система, не требующая вмешательства программиста
после её первичной настройки. Приложение находится в публичном доступе и
доступно к
установке.

В то же время, в работу данного приложения может быть внесён ряд
усовершенствований, возможными направлениями которых являются:
\begin{itemize}
    \item Добавление возможности генерации года не только блокчейна, но и
          других реестров
    \item Добавление в реализацию блокчейна алгоритмов по защите приватности
    \item Улучшение временных характеристик алгоритмов, реализованных на Python
          путём имплементации их на языке CИ
\end{itemize}

К основным направлениям дальнейшей работы над исследовательской частью работы
можно отнести:
\begin{itemize}
    \item Исследование работы внутренней структуры новых реестров
    \item Сравненительный анализ блокчейна и новых реестров
    \item Поддержание разработанной классификации в актуальном состоянии
\end{itemize}



\newpage
\section{Список использованных источников}
\input{used_literature}

\newpage
\section{Приложение А}
\includepdf[pages=-]{tz.pdf}

\newpage
\section{Приложение Б}
\includepdf[pages=-]{ro.pdf}

\newpage
\section{Приложение В}
\includepdf[pages=-]{pimi.pdf}

\newpage
\section{Приложение Г}
\includepdf[pages=-]{tp.pdf}

\end{document}
