В работе были рассмотрены использования современных алгоритмов и протоколов в
распределенных реестрах, была актуализирована их классификация. Наличие такого
объёма алгоритмов по защите конфиденциальности и приватности сигнализирует о
явных проблемах в данной области. И действительно, как показало исследование,
даже такие алгоритмы как Coinshuffle и Stealth Address не гарантируют
анонимность отправителей и получателей. Реализация современных распределённых
реестров предполагает гибкость и адаптируемость под новые алгоритмы и
протоколы, которые появляются каждый месяц. В будущем мы надеемся увидеть
прогресс в области разработки новых (основанных на свежих идеях) алоритмов и
протоколов, что приведёт к скачку и новому всплеску популярности распределённых
реестров в обществе.

Написанная библиотека доступна для свободной установки и помогает разобраться в
практических реализациях написанных алгоритмов, а так же даёт возможность
построить учебную версию распределённого реестра. %(\ref{pril1}).
