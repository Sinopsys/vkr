% В работе были рассмотрены использования современных алгоритмов и протоколов в
% распределенных реестрах, была актуализирована их классификация. Наличие такого
% объёма алгоритмов по защите конфиденциальности и приватности сигнализирует о
% явных проблемах в данной области. И действительно, как показало исследование,
% даже такие алгоритмы как Coinshuffle и Stealth Address не гарантируют
% анонимность отправителей и получателей. Реализация современных распределённых
% реестров предполагает гибкость и адаптируемость под новые алгоритмы и
% протоколы, которые появляются каждый месяц. В будущем мы надеемся увидеть
% прогресс в области разработки новых (основанных на свежих идеях) алоритмов и
% протоколов, что приведёт к скачку и новому всплеску популярности распределённых
% реестров в обществе.

% Написанная библиотека доступна для свободной установки и помогает разобраться в
% практических реализациях написанных алгоритмов, а так же даёт возможность
% построить учебную версию распределённого реестра. %(\ref{pril1}).


Было проведено исследование использования различных алгоритмов и протоколов в
распределённых реестрах. Были найдены новые типы распределённых реестров, такие
как Holochain, Hashgraph и Tempo. Это позволило решить проблему устаревшей
информации по классификации алгоритмов и протоколов в распределённых реестрах.\\
Была разработана актуальная классификация использования алгоритмов и протоколов
в распределённых реестров, в которой отражено современное многообразие
технологичных подходов к решению проблем безопасности.

Было разработано гибкое, масштабируемое программное средство для автоматизации
работы программирования. Разработан уникальный процесс по работе с исходными
кодами алгоритмов, расположенных удалённо, их использованию и автообновлению.
Налажена самоподдерживаемая система, не требующая вмешательства программиста
после её первичной настройки. Приложение находится в публичном доступе и
доступно к
установке.

В то же время, в работу данного приложения может быть внесён ряд
усовершенствований, возможными направлениями которых являются:
\begin{itemize}
    \item Добавление возможности генерации года не только блокчейна, но и
          других реестров
    \item Добавление в реализацию блокчейна алгоритмов по защите приватности
    \item Улучшение временных характеристик алгоритмов, реализованных на Python
          путём имплементации их на языке CИ
\end{itemize}

К основным направлениям дальнейшей работы над исследовательской частью работы
можно отнести:
\begin{itemize}
    \item Исследование работы внутренней структуры новых реестров
    \item Сравненительный анализ блокчейна и новых реестров
    \item Поддержание разработанной классификации в актуальном состоянии
\end{itemize}

