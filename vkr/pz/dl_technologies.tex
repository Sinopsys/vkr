В данной главе описаны возможные технологии распределённых реестров. Под
технологиями понимаются алгоритмы, протоколы, а так же общая структура реестра.
Распределённые реестры делятся на открытые (Public), закрытые (Private),
эксклюзивные (Permissioned), и инклюзивные (Permissionless).
По структуре реализации распределённые реестры делятся на 2 вида: блокчейны и
направленные ациклические графы (DAGs). Алгоритмы, представленные во всех типах
являются неотъемлимой их частью. Это алгоритмы хэширования, электронных
подписей, генерации случайных чисел. Специфические для конкретных реестров
алгоритмы, обеспечивающие должный уровень безопасности/анонимности, например
Ring signatures, Coinjoin, Coinshuffle, stealth addresses, MimbleWimble, etc.
будут рассмотрены в Главе 2. Помимо алгоритмов, будут рассмотрены и различные
протоколы консенсуса, (общей сложностью более 20 штук), обеспечивающие защиту и
надёжность транзакций в открытых блокчейнах.

\subsection{Виды распределённых реестров}\label{kinds_reestrs}
Для разделения реестров на группы по признакам открытости и закрытости
существует несколько подходов.
\begin{itemize}
    \item Первый --- канадский, основанный на публикации статьи [X] создателя
        криптовалюты Ethereum Виталика Бутерина. Автор разделяет 3 типа
        реестров:
          \begin{enumerate}
              \item Публичный (Public), где каждый может принять участие в
                  создании блоков, которое никем не контролируются и
                  выполняется в свободном порядке;
              \item Приватный (Fully Private), где все транзакции отслеживаются
                  и контроллируются централизованной сущностью;
              \item Реестры консорциума (Consortium), где только избранные узлы
                  цепи контроллируют создание новых блоков.
          \end{enumerate}
     \item Второй --- британский. Определения Сэра Марка Уолпорта, главного
         научного советника Соединённого Королевства, в докладе о
         распределённых реестрах [X], совпадают по своей сути с определениями
         Бутерина, но отличаются названиями:
         \begin{enumerate}
                 \item Unpermissioned public;
                 \item Permissioned private;
                 \item Permissioned public.
         \end{enumerate}
     \item Третий --- российский. Часто, для избежания недопониманий и
         сложностей в определении, мировые эксперты используют понятия
         Публичный (Открытый) и Приватный (Закрытый) реестры. Российские
         эксперты не исключение. В своём докладе [X], Ольга Скоробогатова,
         заместитель председателя ЦБ РФ, разделяет реестры именно таким
         образом. Данное разделение будет использоваться в настоящей работе:
         \begin{enumerate}
             \item Публичный (Открытый, Public, Open). В открытых реестрах нет
                 контроллирующей стороны, как это реализовано в закрытых
                 реестрах, все транзакции происходят в свободном порядке, а для
                 подтверждения легитимности транзакции используются специальные
                 протоколы консенсуса (\ref{consensus_protocols})
             \item Приватный (Закрытый, Private, Closed).  Не смотря на то, что
                 сама рассматриваемая технология является распределённой,
                 элементы централизованности присутствуют в таких вариантах,
                 как закрытые (\ref{kinds_reestrs}) реестры. Закрытым реестр
                 может являться благодаря первичному блоку (в случае
                 блокчейна), который будет использоваться. Любой узел может
                 присоединиться к приватному реестру, если он знает адрес
                 начального узла для синхронизации и идентификатор сети. Этот
                 узел может выполнять любые действия в закрытом реестре:
                 майнить, совершать транзакции, заключать контракты, и т.д.\\

                 Такие реестры зачастую используюся фирмами, банками и т.д. для
                 организации внутренних операций обмена и регистрирования.
         \end{enumerate}
\end{itemize}

\subsection{Структура реестров}
По своей внутренней структуре представления распределённые реестры делятся на
блокчейны (\ref{struct_block}) и направленные ациклические графы
(\ref{struct_dags}). На рисунке \ref{graph_reester} изображены виды современных
распределённых реестров.

\begin{figure}[h]
    \Tree [.DL [.DAG ] [.Blockchain ] [.Hybrids\ \&\ Others ]]
    \caption{Виды распределённых реестров}\label{graph_reester}
\end{figure}

\subsubsection{Блокчейны}\label{struct_block}
Блокчейны --- наиболее распространённые сегодня виды реестров. Набор транзакций
(операций) собирается в один блок. Проходит определенное время и этот блок
добавляется в общую цепочку. Для подтверждения транзакции существует множество
протоколов консенсуса --- стандартов, исходя из которых можно говорить, что
создание и добавление данного блока имеет место быть. Подтверждение легальности
добавления блоков в других представлениях распределённых реестров происходит с
приминением других алгоритмов и протоколов, поэтому далее для протоколов
консенсуса будет использоваться слово ``блокчейн'' вместо ``реестр'' для
исключения возможности двоякой интерпретации (на самомм деле, в некоторых
реализациях DAG (X) может применяться протокол консенсуса \ref{pow} для защиты
от спама). Блокчейны завоевали стратегические позиции на площадке
распределённых реестров и ``прошли проверку временем'', но появляются новые
технологии, такие как DAG и другие, призванные решить некоторые недостатки
сетей блокчейн, речь о которых пойдёт позже.

\subsubsection{Направленные ациклические графы (DAGs)}\label{struct_dags}
В DAG все новые записи добавляются в общую цепочку (корректнее --- граф)
асинхронно. История записи операций выглядит как направленный ациклический граф
(wiki). DAG топологически отсортирован так, что каждое ребро направлено от
более раннего ребра к более позднему.

\subsubsection{Плюсы и минусы DAG}
Рассмотрим некоторые преимущества и недостатки DAG по отношению к блокчейнам.\\
Плюсы:
\begin{itemize}
    \item Масштабируемость
    \item Мнгновенные транзакции
    \item Отсутствие либо чрезмерно малые (не заметные) комиссии за переводы
\end{itemize}
Эти плюсы открывают дорогу для огромного количества микротранзакций, делая
систему пригодной, для, например, интернета вещей.\\
Минусы:
\begin{itemize}
    \item Возможные в будущем проблемы с масштабируемостью
    \item Нет подтверждённой учёными информации о защите от взлома основанных на DAG систем
\end{itemize}

\subsection{Алгоритмы}
\subsubsection{SHA-256}
\subsubsection{ECSDA}


\subsection{Особые алгоритмы}
Помимо стандартных криптографических алгоритмов шифрования, генерации
электронной подписи и случайных чисел, в распределённых реестрах повсеместно
используются алгоритмы, обеспечивающие сокрытие персональных данных и адресов
отправителей и получателей, алгоритмы по защите и обфусцированию данных,
отправляющихся по небезопасным каналам связи.
\subsubsection{Ring Signatures}
\subsubsection{Stealth Addressess}
\subsubsection{Coinjoin, coinshuffle}
\subsubsection{Sigma Protocol}
\subsubsection{Zerocash}
\subsubsection{CT, RingCT}
\subsubsection{MimbleWimble}
\subsubsection{Zexe}


\subsection{Протоколы консенсуса}\label{consensus_protocols}
Протокол консенсуса --- стандарт, описывающий правила создания блоков в
открытых распределённых реестрах. В работе рассмотрены все протоколы
консенсуса, используемые в распространённых на сегодняшний день распределённых
реестрах.
\subsubsection{Proof of Work (PoW)}\label{pow}
Алгоритм доказательства работы (Proof of Work) используется в открытых реестрах
для создания новых блоков. Алгоритм хорошо проявляет себя для доказательства
легитимности транзакции и предотвращения ``двойной траты''.  Выполняя это
доказательство, майнеры вознаграждаются процентом, соответствующим
вычислительной сложности всей сети. Зарекомендовавшее себя средство, оно не
является выгодным с точки зрения расхода природных ресурсов, поскольку объём
энергозатрат для выполнения вычислений, необходимых на добавление одного блока,
сопоставим с объемом энергии, потребляемым двумя средними американскими домами (Х).
\subsubsection{Proof of Stake (PoS)}
Алгоритм доказательства доли владения --- основной конкурент предыдущего
алгоритма. Вероятность формирования участником очередного блока в реестре
пропорциональна доле, которую составляют принадлежащие этому участнику
расчётные единицы данного реестра (напр., кол-во единиц криптовалюты) от их
общего количества. Поскольку для принятия решения о том, кто будет являться
создателем нового блока, нет необходимости в большом объёме вычислений, как в
PoW, зачастую является выбором, когда экологический фактор поставлен на одно из
главных мест.
\subsubsection{Delegated Proof-of-Stake (DPoS)}
\subsubsection{Proof-of-Authority (PoA)}
\subsubsection{Proof-of-Weight (PoWeight)}
\subsubsection{Byzantine Fault Tolerance (BFT)}
\subsubsection{Consensus in DAGs}
