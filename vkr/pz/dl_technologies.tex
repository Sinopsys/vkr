В данной главе описаны возможные технологии распределённых реестров. Под
технологиями понимаются алгоритмы, протоколы, а так же общая структура реестра.
Распределённые реестры делятся на открытые (Public), закрытые (Private),
эксклюзивные (Permissioned), и инклюзивные (Permissionless).
По структуре реализации распределённые реестры делятся на 2 вида: блокчейны и
направленные ациклические графы (DAGs). Алгоритмы, представленные во всех типах
являются неотъемлимой их частью. Это алгоритмы хэширования, электронных
подписей, генерации случайных чисел. Специфические для конкретных реестров
алгоритмы, обеспечивающие должный уровень безопасности/анонимности, например
Ring signatures, Coinjoin, Coinshuffle, stealth addresses, MimbleWimble, etc.
будут рассмотрены в Главе 2. Помимо алгоритмов, будут рассмотрены и различные
протоколы консенсуса, (общей сложностью более 20 штук), обеспечивающие защиту и
надёжность транзакций в открытых блокчейнах.


\subsection{Открытые и закрытые реестры}
Public, private.
