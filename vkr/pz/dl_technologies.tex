В данной главе описаны возможные технологии распределённых реестров. Под
технологиями понимаются алгоритмы, протоколы, а так же общая структура реестра.
Распределённые реестры делятся на открытые (Public), закрытые (Private),
эксклюзивные (Permissioned), и инклюзивные (Permissionless).
По структуре реализации распределённые реестры делятся на 2 вида: блокчейны и
направленные ациклические графы (DAGs). Алгоритмы, представленные во всех типах
являются неотъемлимой их частью. Это алгоритмы хэширования, электронных
подписей, генерации случайных чисел. Специфические для конкретных реестров
алгоритмы, обеспечивающие должный уровень безопасности/анонимности, например
Ring signatures, Coinjoin, Coinshuffle, stealth addresses, MimbleWimble, etc.
будут рассмотрены в Главе 2. Помимо алгоритмов, будут рассмотрены и различные
протоколы консенсуса, (общей сложностью более 20 штук), обеспечивающие защиту и
надёжность транзакций в открытых блокчейнах.

% TODO
\subsection{Структура реестров}
\subsubsection{Блокчейны}
\subsubsection{Направленные ациклические графы (DAGs)}


\subsection{Открытые и закрытые реестры}
\subsubsection{Закрытые (private) реестры}
Не смотря на то, что сама рассматриваемая технология является распределённой,
элементы централизованности присутствуют в таких вариантах, как закрытые [X] и
эксклюзивные [X] реестры. Закрытым реестр может являться благодаря первичному
блоку (в случае блокчейна), который будет использоваться. Любой узел может
присоединиться к частному реестру, если он знает адрес начального узла
для синхронизации и идентификатор сети. Этот узел может выполнять
любые действия в частном реестре майнить, совершать транзакции, заключать
контракты, и т.д.

Такие реестры зачастую используюся фирмами, банками и т.д. для организации
внутренних операций обмена и регистрирования.

\subsubsection{Открытые (public) реестры}
В открытых реестрах нет контроллирующей стороны, как это реализовано в закрытых реестрах.

\subsection{Разрешённые реестры}
\subsubsection{Эксклюзивные (permissioned) реестры}
\subsubsection{Инклюзивные (permissionless) реестры}


\subsection{Алгоритмы}
\subsubsection{SHA-256}
\subsubsection{ECSDA}


\subsection{Особые алгоритмы}
\subsubsection{Ring Signatures}
\subsubsection{Stealth Addressess}
\subsubsection{Coinjoin, coinshuffle}
\subsubsection{Sigma Protocol}
\subsubsection{Zerocash}
\subsubsection{CT, RingCT}
\subsubsection{MimbleWimble}
\subsubsection{Zexe}


\subsection{Протоколы консенсуса}
\subsubsection{Proof of Work (PoW)}
Алгоритм доказательства работы (Proof of Work) используется в открытых реестрах
для создания новых блоков. Алгоритм хорошо проявляет себя для доказательства
легитимности транзакции и предотвращения ``двойной траты''.  Выполняя это
доказательство, майнеры вознаграждаются процентом, соответствующим
вычислительной сложности всей сети. Зарекомендовавшее себя средство, оно не
является выгодным с точки зрения расхода природных ресурсов, поскольку объём
энергозатрат для выполнения вычислений, необходимых на добавление одного блока,
сопоставим с объемом энергии, потребляемым двумя средними американскими домами (Х).
\subsubsection{Proof of Stake (PoS)}
Алгоритм доказательства доли владения --- основной конкурент предыдущего
алгоритма. Вероятность формирования участником очередного блока в реестре
пропорциональна доле, которую составляют принадлежащие этому участнику
расчётные единицы данного реестра (напр., кол-во единиц криптовалюты) от их
общего количества. Поскольку для принятия решения о том, кто будет являться
создателем нового блока, нет необходимости в большом объёме вычислений, как в
PoW, зачастую является выбором, когда экологический фактор поставлен на одно из
главных мест.
\subsubsection{Delegated Proof-of-Stake (DPoS)}
\subsubsection{Proof-of-Authority (PoA)}
\subsubsection{Proof-of-Weight (PoWeight)}
\subsubsection{Byzantine Fault Tolerance (BFT)}
\subsubsection{Consensus in DAGs}
