\documentclass[russian, a4paper, 12pt]{article}
\usepackage[utf8]{inputenc}
\usepackage[russian]{babel}
\usepackage{setspace}
\usepackage[left=20mm, top=15mm, right=15mm, bottom=15mm, nohead, footskip=10mm]{geometry} % настройки полей документа
\usepackage{tocloft}
\usepackage{hyperref}
\hypersetup{
    colorlinks,
    citecolor=black,
    filecolor=black,
    linkcolor=black,
    urlcolor=black
}

%%% Работа с русским языком
\usepackage{cmap}					% поиск в PDF
\usepackage[T2A]{fontenc}			% кодировка
\usepackage[utf8]{inputenc}			% кодировка исходного текста

%%% Повтор команды несколько раз через     \Repeat{n-times}{\command}
\usepackage{expl3}
\ExplSyntaxOn
\cs_new_eq:NN \Repeat \prg_replicate:nn
\ExplSyntaxOff


\begin{document}
\thispagestyle{empty}
{\Large
\begin{center}
    Отзыв руководителя о прохождении производственной практики\\
    студентом группы БПИ 151 образовательной программы «Программная инженерия»\\
    факультета компьютерных наук НИУ ВШЭ\\
    Куприянова Кирилла Игоревича.
\end{center}
}

\noindentВо время практики студент работал над ВКР на тему
\underline{\hspace{7cm}}\\\\
\underline{\hspace{\textwidth}}\\\\
Период прохождения практики: с 1.04.2019 по 28.04.2019.
    \Repeat{1}{\hfill \break}
%
    С отчетом по практике студента группы БПИ 151 Куприянова Кирилла Игоревича ознакомлен.
    В программе практики перед студентом были поставлены задачи:
    \begin{itemize}
        \item Выявить популярные распределённые реестры и выделить криптографические алгоритмы в них
        \item Изучить выявленные алгоритмы и способы их классификации
        \item Замерить параметры алгоритмов
        \item Классифицировать их согласно выявленным метрикам
        \item Создать обновлённую на 2019 год классификацию алгоритмов и
              протоколов в распределённых реестрах
        \item Реализовать Python3.6.5 библиотеку, содержащую все изученные
              алгоритмы и протоколы, позволяющую создавать распределённый реестр
              в учебных целях
        \item Опубликовать библиотеку в PyPi
    \end{itemize}

    \Repeat{8}{\hfill \break}

Задание на практику выполнено в полном объеме. Студент Куприянов Кирилл Игоревич заслуживает оценки
\underline{\hspace{0.7cm}} баллов из 10
\newline
\newline
Должность \underline{\hspace{15cm}}\\
\newline
Подпись \underline{\hspace{6cm}}
ФИО \underline{\hspace{8.3cm}}
\newline
\newline
Дата \underline{\hspace{4cm}}
\end{document}

% EOF
