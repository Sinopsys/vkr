\subsection{Терминология}
\begin{description}
    \item[Активность (Activity)] --
        Activity — это компонент приложения, который выдает экран, и с которым
        пользователи могут взаимодействовать для выполнения каких-либо
        действий, например набрать номер телефона, сделать фото, отправить
        письмо или просмотреть карту. 

    \item[Фрагмент (Fragment)] -- 
        Фрагмент (класс Fragment) представляет поведение или часть
        пользовательского интерфейса в операции (класс Activity). В одной
        активности может быть несколько фрагментов.

    \item[RecyclerView] -- 
        Элемент визуальной группы для отображения списков объектов,
        переиспользующий место на экране устройства. То есть хранится столько
        объектов, сколько может уместиться на экране, а при прокрутке элемента,
        существующие объекты перерисовываются, а не продолжаются хранится в
        памяти.

    \item[Shared Preferences] -- 
        Файл в системе, содержащий информацию в виде пар ``ключ - значение''.

    \item[Toolbar] --
        Верхняя часть приложения, содержащая наиболее часто используемые
        функциональные элементы, например кнопка ``назад'' или ``создать''.

    \item[Bottom Navigation View] -- 
        Элемент навигации, представляющий из себя несколько кнопок, ведущих на
        разные фрагменты и расположенный снизу экрана устройства.

    \item[Xpath selectors] -- 
        Язык для выделения из *ML (XML, HTML) кода его элементов, тэгов и аттрибутов.

    \item[Cron job] -- 
        Сервис в системах GNU/Linux и UNIX, позволяющий ставить на
        повторяющиеся циклы исполнение пользовательских команд.

    \item[crawler] --
            Программный модуль, работающий в фоне и производящий сбор данных с сайтов
            указанных магазинов, с последующей отправкой их на сервер в формате
            JSON

    \item[Пользовательский товар] --
            Товар, представленный в виде текста, имеющий в себе массив товаров,
            подходящих при сопоставлении названий к данному. Пример.
            Пользоваельский товар ``Сок'' имеет массив сопоставившихся товаров [Сок
            Добрый 1л Яблоко; Сок J-7 апельсин с мякотью].

    \item[Пользовательская сессия, user session] --
           Начинается с момента входа пользователя в систему (log in), и завершается при выходе (log out).

    \item[Spider] --
        Часть crawler'a, отвечающая за непосредственный сбор информации с
        веб-страниц, переход между страницами и дальнейшую отправку собранных
        данных другим модулям crawler'a.

    \item[Non-breaking space] -- 
        Специальный пробельный символ, предотвращающий автоматический разрыв
        строки в месте, где он стоит.

\end{description}

