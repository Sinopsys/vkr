\paragraph{Язык программирования}
Использованный язык программирования --- \textbf{Python} верисии
\underline{3.6.5}.\\
\emph{Рассматриваемые аналоги}: C, Java\\\\
Выбран вследствие своей универсальности применения
относительно (а) алгоритмов, (б) платформы для запуска; а так же простоты
реализации побочных, вспомогательных компонент, посредственно относящихся к
данному проекту. Реализация их на других языках была бы необходимостью --- и, в
следствии их посредственного отношения к проекту, необоснованной тратой
временного ресурса.\\
\emph{Реализации алгоритмов \textbf{keccak-256 \emph{ и } keccak-512} были
переписаны с Python2 на Python3.6.5 для совместимости с данной программой}.

\paragraph{Тип приложения}
Приложение является консольной утилитой, которая может быть установлена в
систему семейства \underline{GNU/Linux} при помощи программы
\textbf{python3-pip}.\\
\emph{Рассматриваемые аналоги}: Приложение с графическим интерфейсом,
приложение в веб-интерфейсом\\\\
Отсутствие графического интерфейса обосновано отсутствием
необходимости произведения манипуляций при помощи мыши, и отсутствием
необходимости отображения графических изображений и другой информации.
Интерактивный диалог производится посредством вывода в стандартный выход
(\emph{stdout}) консоли текста с опциями; а выбор пользователя регистрируется
посредством считывания стандартного ввода (\emph{stdin}).

\paragraph{Протокол обмена данными между компонентами}
Выбранной опцией является передача \textbf{json} файлов посредством \textbf{http} протокола.\\
\emph{Рассматриваемые аналоги}: \textbf{grpc}
(http://flask.pocoo.org/docs/0.12/api/\#flask.Flask), \textbf{zmq}
(http://zeromq.org/)\\\\
Обеспечивается использованием модуля \emph{Flask} и \emph{json}. В сравнении
с \emph{grpc} (https://grpc.io/) и \textit{zmq} протоколами, \emph{http}
представлялся наиболее подходящим вследствие своей популярности и удобства
подключения и использования совместно с языком Python.

\paragraph{Хранилище}
В качестве хранилища было реализовано \textbf{key-value} (ключ-значение) хранилище.\\
\emph{Рассматриваемые аналоги}: \textbf{etcd} (https://coreos.com/etcd/), \textbf{sqlite} (https://sqlite.org/index.html)\\\\
Подключение сторонней библиотеки или базы данных сильно увеличило бы вес
приложения в целом, а так же добавило бы ещё несколько зависимостей. Вследствие
этого было решено придумать импровизированное key-value хранилище на основе
структуры данных словарь (dictionary).
Реализация и функциональность описана в настоящем техническом задании
(Приложение REF).

\paragraph{Автообновление}
Задачей автообновления занимается UNIX утилита \textbf{cron}.\\
\emph{Рассматриваемые аналоги}: Импровизированный планировщик событий\\\\
В связи с существованием качественного и надёжного решения в лице
\textbf{cron}'a, было решено использовать его. Настроен на сервере. Подробнее
--- в п. \ref{autoonova}

\paragraph{Continuous integration}
В качестве средства CI был выбран \textbf{Shippable} (app.shippable.com).\\
\emph{Рассматриваемые аналоги}: \textbf{TravisCI} (https://travis-ci.org/), \textbf{CircleCI} (https://circleci.com/)\\\\
Данное средство распространяется с возможностью использовать бесплатную версию
программы, поэтому выбор пал на неё. \emph{Необходимость} в сервисе CI
выражается в том, чтобы данное приложение генерировало рабочие стабильные коды
даже после обновления используемых реализаций алгоритмов, что не зависит от
разработчика.

\paragraph{Конфигурация}
Конфигурационный файл программы написан на языке \textbf{YAML}.\\
\emph{Рассматриваемые аналоги}: \textbf{JSON}, \textbf{Plain text}\\\\
Выбран \emph{YAML} по причине хорошей поддерживаемости в Python и своей эстетической
непритязательности по сравнению с аналогами.


