Реализация блокчейна делится на 2 модуля: {\small wallet.py} и {\small
miner.py}. Описание каждого из них представлено в данной секции.
Данные модули генерируются компоновщиком, описанным в п. \ref{komponovshik}.
Компоновщик сгенерировал данные модули таким образом, что они используют
реализации алгоритмов, выбранных пользователем.

\subsubsection{wallet.py}
Модуль, позволяющий выполнить по запуску со стороны пользователя одну из трёх
функциональностей:

\begin{enumerate}
    \item Создать новый кошелёк
    \item Отправить средств
    \item Валидировать транзакции
\end{enumerate}


Модуль должен использовать алгоритмы, выбранные пользователем на этапе
генерации. Поэтому в начале, алгоритм импортирует их (листинг
\ref{lst:import}). В противном случае, при невозможности их импортирования,
модулем будут применены стандартные алгоритмы.  Подобными для хэширования
является \textbf{SHA-256}, а для цифровой подписи \textbf{ECSDA}. 

\begin{center}
\begin{lstlisting}
try:
    import mydss
    dss = mydss
    if hasattr(dss, 'name') and hasattr(dss, 'bit'):
        alg_name = dss.name
        alg_bit = dss.bit
        try:
            from mydss import mydss
            dss = mydss
        except:
            dss = mydss
except:
    import ecdsa
    dss = ecdsa
    alg_name = 'ecdsa'
    alg_bit = '256'
\end{lstlisting}\label{lst:import}
    Листинг \ref{lst:import}: Импортирование выбранных пользователем опций (myhashing и mydss)
\end{center}

\paragraph{Создание нового кошелька}
При создании нового кошелька, запрашивается имя пользователя, и генерируются
публичный и приватный ключи. Далее, они записываются в файл
\textbf{<name>.txt}.  Генерация ключей осуществляется в методе {\small
generate\_ECDSA\_keys(ret=False)}.  При генерации по умолчанию, используется
алгоритм \textbf{ECSDA}. При успешной попытке использования указанного
пользователем алгоритма, используется он. Для компактности записи, длинный
публичный ключ кодируется в base64 и в таком формате записыватеся в файл. В
дальнейшем, он будет декодирован для корректной процедуры отправки/верификации.
Описанная процедура представлена в листинге \ref{lst:creation}.

\begin{center}
\begin{lstlisting}
def generate_keys(ret=False):
    if _timed:
        t1 = time.time()
    singningkey = dss.SigningKey().generate()
    private_key = singningkey.to_string()
    vk = singningkey.getverifyingkey()
    public_key = vk.to_string().hex()
    public_key = singningkey.to_string(pub=True)
    if _timed:
        t2 = time.time()
        _write_time(alg_name, 'Key pair generation', alg_bit, t2-t1)
    public_key = base64.b64encode(bytes.fromhex(public_key))
    if ret:
        return private_key, public_key.decode()
    filename = input('Name of addr: ') + '.txt'
    with open(filename, 'w') as f:
        f.write('PrvK: {0}\nWallet addr / PubK: {1}'.format(private_key, public_key.decode()))
    print('saved{0}'.format(filename))
\end{lstlisting}\label{lst:creation}
    Листинг \ref{lst:creation}: Генерация пары публичного и приватного ключей
\end{center}



\paragraph{Отправка средств}
При отправки средств, используются стандартные подходы блокчейна. В модуле
{\small miner.py} происходит (Листинг \ref{lst:send}) основная обработка и
добавление самого блока к общей цепочке. В рассматриваемом {\small wallet.py},
происходит лишь сбор всей необходимой для совершения транзакции в json payload
и отправляется на URL в API модуля {\small miner.py}. Дальнейшая обработка
отправленного запроса происходит в \ref{lst:newblock}.

\begin{center}
\begin{lstlisting}
def _perform_transaction(from_, prv_key, addr_to, amount):
    len_prv = len(prv_key)

    if dss.name == 'gost' or len_prv == 64:
        signature, message = _sign_msg(prv_key)
        url = f'http://localhost:{_port}/mycoin'
        payload = {'from': from_,
                   'to': addr_to,
                   'amount': amount,
                   'signature': signature.decode(),
                   'message': message}
        headers = {'Content-Type': 'application/json'}

        res = requests.post(url, json=payload, headers=headers)
        print(res.text)
    else:
        print('Wrong address; Please try again.')
\end{lstlisting}\label{lst:send}
    Листинг \ref{lst:send}: Отправка сформированного блока в {\small miner.py}
\end{center}


\paragraph{Валидация трназакций}
Валидация транзакций в скрипте {\small wallet.py} ограничивается отправкой в
{\small miner.py} запроса на проверку блоков (Листинг \ref{lst:valid}).

\begin{center}
\begin{lstlisting}
    def check_transactions():
        res = requests.get(f'http://localhost:{_port}/blocks')
        print(res.text)
\end{lstlisting}\label{lst:valid}
    Листинг \ref{lst:valid}: Отправка запроса в {\small miner.py} на проверку
    валидности транзакций
\end{center}


\subsubsection{miner.py}
Данный модуль запускается и работает как фоновый процесс во время отправки и
валидации сообщений первого. Он выполняет такие операции как mine block (майннг
блока) --- процесс, в котором происходит вызов proof-of-work процедуры, после
которой считается возможным добавить в цепочку блок.

Со стороны http api для {\small wallet.py}, он выполняет:
\begin{enumerate}
    \item Принимает \textbf{POST} запрос на добавление нового блока в цепочку
    \item Принимает \textbf{GET} запрос на проверку существующих блоков в
          цепочке
\end{enumerate}

\paragraph{Создание нового блока}
Создание нового блока происходит в несколько этапов.

Информация о входящей транзакции регистрируется в очереди {\small
NODE\_PENDING\_TRANSACTIONS} и выводится на стандартный вывод (stdout) ---
Листинг \ref{lst:newblock}.  Процесс, запущенный в отдельном потоке исполнения
ОС, целевой функцией которого является mine, получает изменения из очереди
{\small NODE\_PENDING\_TRANSACTIONS} и запускает процесс майна нового блока ---
Листинг \ref{lst:mine}.

\begin{center}
\begin{lstlisting}
    @app.route('/mycoin', methods=['GET', 'POST'])
    def transaction():
        """Each transaction sent to this node gets validated and submitted.
        Then it waits to be added to the blockchain. Transactions only move
        coins, they don't create it.
        """
        if request.method == 'POST':
            new_mycoin = request.get_json()
            if validate_signature(new_mycoin['from'], new_mycoin['signature'], new_mycoin['message']):
                WAITING_TRANSACTIONS.append(new_mycoin)
                print("Got a new transaction")
                print("FROM: {0}".format(new_mycoin['from']))
                print("TO: {0}".format(new_mycoin['to']))
                print("AMOUNT: {0}\n".format(new_mycoin['amount']))
                return "Transaction submission successful\n"
            else:
                return "Transaction submission failed. Wrong signature\n"
        elif request.method == 'GET' and request.args.get("update") == MINER_ADDRESS:
            pending = json.dumps(WAITING_TRANSACTIONS)
            WAITING_TRANSACTIONS[:] = []
            return pending
\end{lstlisting}\label{lst:newblock}
    Листинг \ref{lst:newblock}: URL '/txion' в {\small miner.py} для создания нового блока
\end{center}


\begin{center}
\begin{lstlisting}
def mine(a, blockchain, WAITING_TRANSACTIONS):
    BLOCKCHAIN = blockchain
    WAITING_TRANSACTIONS = WAITING_TRANSACTIONS
    while True:
        if _timed:
            t1 = time.time()
        last_block = BLOCKCHAIN[len(BLOCKCHAIN) - 1]
        last_proof = last_block.data['proof-of-work']
        proof = proof_of_work(last_proof, BLOCKCHAIN)
        if not proof[0]:
            BLOCKCHAIN = proof[1]
            a.send(BLOCKCHAIN)
            continue
        else:
            WAITING_TRANSACTIONS = requests.get(MINER_NODE_URL + "/txion?update=" + MINER_ADDRESS).content
            WAITING_TRANSACTIONS = json.loads(WAITING_TRANSACTIONS)
            WAITING_TRANSACTIONS.append({
                "from": "network",
                "to": MINER_ADDRESS,
                "amount": 1})
            new_block_data = {
                "proof-of-work": proof[0],
                "transactions": list(WAITING_TRANSACTIONS)
            }
            WAITING_TRANSACTIONS = []
            new_block_index = last_block.index + 1
            new_block_timestamp = time.time()
            last_block_hash = last_block.hash
            mined_block = Block(new_block_index, new_block_timestamp, new_block_data, last_block_hash)
            BLOCKCHAIN.append(mined_block)
            try:
                print(json.dumps({
                  "index": new_block_index,
                  "timestamp": str(new_block_timestamp),
                  "data": new_block_data,
                  "hash": last_block_hash.decode()
                }) + "\n")
            except:
                print(json.dumps({
                  "index": new_block_index,
                  "timestamp": str(new_block_timestamp),
                  "data": new_block_data,
                  "hash": last_block_hash
                }) + "\n")
            a.send(BLOCKCHAIN)
            requests.get(MINER_NODE_URL + "/blocks?update=" + MINER_ADDRESS)
            if _timed:
                t2 = time.time()
                _write_time(hash_name, 'Mining one block', hash_bit, t2-t1)
\end{lstlisting}\label{lst:mine}
    Листинг \ref{lst:mine}: Процесс майнинга новых блоков
\end{center}


\paragraph{Валидация подписи}
Валидация электронной подписи происходит в методе {\small validate\_signature}.
В нём вызываются методы алгоритма цифровой подписи. Алгоритм может быть либо
выбранный пользователем на этапе генерации, или, если его не происходило,
стандартным, то есть \textbf{ECSDA} с кривой  SECP256k1 --- Листинг
\ref{lst:validate}.


\begin{center}
\begin{lstlisting}
def validate_signature(public_key, signature, message):
    if _timed:
        t1 = time.time()
    public_key = (base64.b64decode(public_key)).hex()
    signature = base64.b64decode(signature)
    vk = dss.VerifyingKey().from_string(public_key)
    try:
        res = vk.verify(signature, message.encode())
        if _timed:
            t2 = time.time()
            _write_time(alg_name, 'Verifying signature', alg_bit, t2-t1)
        return res
    except:
        return False
\end{lstlisting}\label{lst:validate}
    Листинг \ref{lst:validate}: Процесс верификации электронной подписи
\end{center}
