\subsection{Загрузка файлов}
В данном разделе будет показано, как устанавливать программу и пользоваться ей. Для описания каждого проделанного шага будут включены описание действий и скриншоты приложения.
\subsubsection{Установка программы}
Дистрибутив данной программы можно будет получить по ссылке, считав ее с
прилагаемого qr-кода(рис.~\ref{ref})

\begin{figure}[h!]
    \centering
    % \includegraphics[width=0.7\textwidth]{./screenshots/3/qr.png}
    \caption{ссылка на программу}
    \label{ref}
\end{figure}

Чтобы начать испытания выполнения требований к функциональным характеристикам,
необходимо запустить установщик программы путем нажатия на его иконку. После
запуска установщика появится приветственное окно мастера установки (рис.~\ref{install1},~\ref{install2},~\ref{install3}).

\begin{figure}[h!]
    \centering
    \minipage{0.3\textwidth}
    % \includegraphics[height=0.38\textheight]{./screenshots/3/install_1.jpg}
    \caption{\small{начало установки}}
    \label{install1}
    \endminipage\hfill
    \minipage{0.3\textwidth}
    % \includegraphics[height=0.38\textheight]{./screenshots/3/install_2.jpg}
    \caption{\small{процесс установки}}
    \label{install2}
    \endminipage\hfill
    \minipage{0.3\textwidth}
    % \includegraphics[height=0.38\textheight]{./screenshots/3/install_3.jpg}
    \caption{\small{приложение установлено}}
    \label{install3}
    \endminipage{}
\end{figure}


\subsubsection{Использование приложения}
Затем приложение необходимо открыть. Оператора будет приветствовать экран входа
в аккаунт. Есть возможность использовать приложение без входа и регистрации, но
при таком условии, оператору будут доступен ограниченный функционал, а именно
только просмотр акций. Регистрация даёт возможность работы со списками покупок.
(рис.~\ref{login},~\ref{register}).

\begin{figure}[h!]
    \centering
    \minipage{0.45\textwidth}
    % \includegraphics[height=0.37\textheight]{./screenshots/3/login.jpg}
    \caption{\small{вход в аккаунт}}
    \label{login}
    \endminipage\hfill
    \minipage{0.45\textwidth}
    % \includegraphics[height=0.37\textheight]{./screenshots/3/register.jpg}
    \caption{\small{регистрация аккаунта}}
    \label{register}
    \endminipage{}
\end{figure}

Войдя в аккаунт или выбрав режим без регистрации, пользователь попадает на
главный экран приложения (рис.~\ref{home}).

\begin{figure}[h!]
    \centering
    % \includegraphics[height=0.35\textheight]{./screenshots/3/home.jpg}
    \caption{\small{просмотр всех акций}}
    \label{home}
\end{figure}

Далее, из любой активности оператор может вызвать раздел help (рис. \ref{help})
в меню приложения, где указана инструкция пользования данным продуктом. Также,
оператор имеет возможность просмотреть раздел about (рис. \ref{about}), где
написана общая информация о приложении и о разработчиках, и по долгому зажатию
на элементы button (кнопки), узнать её функционал (\ref{long_click}), что поможет ему лучше
разбираться в работе приложения.

\begin{figure}[h!]
    \centering
    \minipage{0.3\textwidth}
    % \includegraphics[height=0.35\textheight]{./screenshots/3/about.jpg}
    \caption{\small{просмотр раздела ``о приложении''}}
    \label{about}
    \endminipage\hfill
    \minipage{0.3\textwidth}
    % \includegraphics[height=0.35\textheight]{./screenshots/3/help.jpg}
    \caption{\small{просмотр подсказки пользования приложением}}
    \label{help}
    \endminipage\hfill
    \minipage{0.3\textwidth}
    % \includegraphics[height=0.35\textheight]{./screenshots/3/hint.jpg}
    \caption{\small{просмотр всплывающего описания элементов button (кнопок)}}
    \label{long_click}
    \endminipage{}
\end{figure}

\newpage

