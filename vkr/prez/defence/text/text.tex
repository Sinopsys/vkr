%!TEX TS-program=xelatex
%!USE flag=shell-escape
\documentclass[12pt]{article}
\usepackage[english,russian]{babel}
\usepackage{fontspec}
\defaultfontfeatures{Ligatures={TeX},Renderer=Basic}
\setmainfont[Ligatures={TeX,Historic}]{Myriad Pro}
\setsansfont{Myriad Pro}
\setmonofont{Courier New}
\usepackage[top=2cm, bottom=2cm, left=2cm, right=2cm]{geometry}

%%%%%% My commands %%%%%%%
%
\renewcommand{\line}{\noindent\rule{\textwidth}{1pt}}
%

\begin{document}

\section*{Текст презентации}

Добрый день, уважаемая комиссия, студенты. Меня зовут Куприянов Кирилл, и я
занимаюсь исследованием криптографическиз алгоритмов и протоколов дл
яраспределенных реестров.\\ Мой научный руководитель --- Авдошин Сергей
Михайлович.\\
\line\
Моя предметная область связана с популярной сегодня темой --- криптовалюты,
распределённый интернет, технология блокчейн, криптография. Очень много работа сейчас публикуется на эту тему, и с момента взятия темы сама область уже расширилась и нужно держаться постояно в курсе чтобы не отстатб от времени.\\
\line\
В процессе своего выступления я буду использовать технические термины,
которые стоит сразу обозначить.\\
\line\
Оглавление ВКР, в главе 1 описана теоретическая база и описание рассмотренных технологий распределённых реестров. Это включает в себя алгоритмы и протоколы, и т.д..
В главе 2 проводится их анализ и сторится классификация, а в третей описывается построение программы.\\
\line\
Актуальность обоснована популярностью технологии блокчейн, отстутствием полной
классификации криптографических алгоритмов, использующихся для функционирования
криптовалют, и необходимостью на сегодняшний день создать такую
классификацию. Люди опираются на бизнес логику а не на реальные факты о реализации и надёжности, так же отсутствует библиотека/программа где всё посмотреть легко.\\
\line\
Вот так выглядит существующая классификация на сегодняшний день, она приведена в книжке [15] Она хорошая и достаточно полной была на 2014 год, но сейчас нада новое\\
\line\
Соответстно, я задался целью проделать анализ криптографических алгоритмов для
популярных в мире распределённых реестров, и для этого мне необходимо было
выявить какие алгоритмы используются в распределенных реестрах, сравнить их по
времени работы и объему памяти, выявить лучших. Это для анализа. А те
алгоритмы, которые я собрал, исследуя распределенные реестры, я планирую
объединить в одну общую библиотеку, так сказать, некий тулкит для
эксперементирования c распределённым реестром.\\
\line\
Все распределённые реестры DL делятся на несколько типов на рисунке.\\
\line\
Реестры по открытости бывают рызные. Зависит от того, какой клас-ции
придерживаться. Мы будем российской. Цитаты: 17, 7, 19\\
\line\
В них используются алгоритмы и протоколы и это не полный список, руководствуясь
поисками и наводками я искал и изучал информацию, составлял сравнительные
таблицы и в итоге кое-что получилось\\
\line\
2 ПРОПУСКА СЛАЙДОВ СКАЖУ САМ. Для качественного сравнения нужно было найти
необходимые ресурсы на каждый рассматриваемый реестр. Часть таблицы с
источниками информации приведена на слайде.\\ Дльше сам тоже
\end{document}


% EOF

