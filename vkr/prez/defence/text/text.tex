%!TEX TS-program=xelatex
%!USE flag=shell-escape
\documentclass[12pt]{article}
\usepackage[english,russian]{babel}
\usepackage{fontspec}
\defaultfontfeatures{Ligatures={TeX},Renderer=Basic}
\setmainfont[Ligatures={TeX,Historic}]{Myriad Pro}
\setsansfont{Myriad Pro}
\setmonofont{Courier New}
\usepackage[top=2cm, bottom=2cm, left=2cm, right=2cm]{geometry}

%%%%%% My commands %%%%%%%
%
\renewcommand{\line}{\noindent\rule{\textwidth}{1pt}}
%

\begin{document}

\section*{Текст презентации}
Текущее  десятилетие --- интересное время развития децентрализованных
технологий. Добрый день, уважаемая комиссия и коллеги-студенты, моя работа,
которую я выполнял под руководством Сергей Михайловича, связана с
криптографическими алгоритмами и протоколами для распределённых реестров.
Стоит отметить заранее, что в работе не предлагаются реализации новых, доселе
неизвестных алгоритмов. Поскольку благодаря усилиям,  которые  на  протяжении
предыдущих тридцати  лет  прикладывали криптографы,  математики  и кодировщики,
разрабатывая  строго  специальные усовершенствованные  протоколы для  защиты
конфиденциальности  и гарантий аутентичности  различных  систем, дают
возможность современным технологиям, утилизируя вычислительные мощности и
применяя знания, усердно накопленные предыдущими поколениями, создавать
необычайные распределённые системы.\\
\line\
Котрые приносят немалый доход. На данном графике виден текущий и будущий
стабильный рост выручки в индустриях где применяется блокчейн.\\
\line\
На создание своей работы я был вдохновлён исследованием Тима Сэунсона,
известной в кругах по интересу блокчейна личности. 


\end{document}


% EOF

