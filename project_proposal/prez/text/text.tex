%!TEX TS-program=xelatex
%!USE flag=shell-escape
\documentclass[12pt]{article}
\usepackage[utf8]{inputenc}
\usepackage[english,russian]{babel}
\usepackage[top=2cm, bottom=2cm, left=2cm, right=2cm]{geometry}

%%%%%% My commands %%%%%%%
%
\renewcommand{\line}[1]{\noindent\\{#1}\vspace{-0.42cm}\\\rule{\textwidth}{1pt}\\}
%

\begin{document}

\section*{Prezentation text}

\line{Introduction}
Good afternoon, everyone. My name is Kirill Kupriyanov and today's topic is
my research proposal for the thesis Cryptographic Algorithms and Protocols for
Distributed Ledgers. This theme is strongly related to cryptographic algorithms,
cryptocurrencies and a blockchain. PAUSE 2 SEC.

\line{Google trends pic}
In recent years the popularity of blockchain has been decreasing steadily,
which can be seen on the line graph presented. There is no comparison between
the spike in 2017 and today's numbers. Still, it remains a (relatively)
frequent topic of discussion. Today, scientists focus mainly on the upcoming
events: overflow of blocks number, higher fees, expanding and popularizing
knowledge of cryptocurrencies in public. That is why we expect to see another
burst of interest and attention in the near future.
That is why we chose this theme, outline of which will be introduced now.

\line{Outline}
Firstly, the study area we are working with, and technichal terms that may be
needed to fully get the content will be introduced. PAUSE 2 SEC. In the main
part important ideas and concepts will be listed. And finally, we will be able
to dicuss questions that might appear during the presentation. Let's move to
the next part.
% Firstly, the study area we are working with, and technichal terms that may be
% needed to fully get the content will be introduced. PAUSE 2 SEC.  Next, a list
% of problems that will be tackled during the future research will be stated.
% Important ideas and concepts will be pointed out in the methods section.
% Expected outcome of the work as well as links and sources that were used for
% preparing a project proposal will also be provided in the end. And finally, we
% will be able to dicuss questions that might appear during the presentation.
% Let's move to the next part.

\line{Study area}
The theme of the research is bound to cutting edge areas. Although
cryptography is a science which has been evolving throughout humanity's
history, now, when most it's methos are known and studied, the interesting part
is applications of this study area.

\line{Cryptobytes}
It applies to every field of Computer Science, because Computer Science is all
about transmitting bytes from one source to another, security of this process
is vital.

\line{Study area}
In the scope of the research the majority of the work will be studying,
analyzing and classifying things, and programming job will also take place.
The Distributed Ledger is the most important part, which should firstly be
defined.

% The blockchain technology's potential to open the door to other
%         revolutionary possibilities

\line{Definitions}
There is a subtle difference between a blockchain and distributed ledger.

\line{Apple slide}
A blockchain is a type of distributed ledger. Much like an apple is a type of
fruit. Apple is a fruit, and a Blockchain is a Ditstributed Ledger. This is a
common confusion, which arises because in most papers the term ``Blockchain''
was introduced before ``Distributed Ledger''. The sudden surge of popularity
had the term ``Blockchain'' turn into a generic term. Furthermore, it became so
generic that most people belive that all cryptocurrencies are blockchains. This
leads to the next confusion.

\line{Blockchain and cryptocurrencies}
Blockchain is just one of the bases for building a cryptocurrency. On the other
side, people invented cryptocurrencies which are based on another type of a
distributed ledger called DAGs (Directed Accyclic Graphs) -- it is a complex
mathematical data structure, concepts of which are not covered in this
presentation. So, which cryptocurrencies are blockchains, and which are DAGs?

\line{Blockchain vs DAGs}
It is important that there are more types of Distributed Ledgers, and more
types of cryptocurrencies, and only some of them are presented on the slide.
NAME OR NOT??? Let's move to the next part of presentation.

\line{Problems}
We detected a few problems in the field. The first problem is that the
classification of all types of algorithms and protocols in Distributed Ledgers
is outdated. Users who want to want to know how a particular distributed ledger
is built, face the need in the fresh classification.

\line{Outdated classification}
Many new cryptocurrencies added, protocols used, and algorithms applied.
Because this classification was designed in 2014, there is a vast area of
upgrading this scheme.

\line{Problems}
The second problem is lack of technical information. The majority of websites
regarding these technologies are consumer oriented, they do not provide
important technichal details, which hardens the search of how a particular
ledger is implemented.\\
The third problem is lack of affordable solutions being able to build a ledger
from scratch, in educational or commercial purposes.\\
We aim to solve all these problems using the following methods.

\line{Methodology of problem 1}
They exactly match the problems. Outdated classification is solved by
analyzing, classifying and structurizing information from trusted sources.

\line{Methodology of problem 2}
The second problem will be tackeled by building a structured, robust, easy to
acces, and holding all needed without unnecessary information in one place,
wiki, like a github, or notabug wiki.\\
The final problem will be solved by producing a Python library, which will contain
all revied algorithms and protocols, and will be published to PyPi and will be
available to everyone in world. The library should be easy to use, well
documented and have a wiki page on github or notabug. This is the last method,
and it is time to conclude the expected results.

\line{Expected results}
In the end of the research and programming work it is expected that all
problems are solved using the stated methods. We expect to have an updated
classification for 2019 and a well-documented code and wiki. The text has to be
written in affordable language without unnecessary information. This will help
people to rapidly find technical info on any algorithm or protocol used in a
particular Distributed Ledger. Let us move to the final.

\line{References}
The references which were used during the preparation are available on the
slide.

\line{Questions}
Now there will be some time for us to discuss questions and unclear moments, if
any. And if we will not succeed to cover all of them, please do not hesitate
contacting us via these contacts.

\line{Thnaks for attentions}
Thanks.

\end{document}


% EOF

