\documentclass[12pt]{article}
\usepackage[utf8]{inputenc}
\usepackage[english,russian]{babel}
\usepackage[top=1in, bottom=1in, left=1in, right=1in]{geometry}
\usepackage{hyperref}

\title{Cryptographic Algorithms and Protocols for Distributed Ledgers}
\author{Kupriyanov K.I.\\Avdoshin S.M.}
\date{January 2019}

\begin{document}

\maketitle
\begin{enumerate}
    \item Abastract
    \item Introduction --- 333 w
    \item Technical Aspects --- 238 w
    \item Problem Statement --- 275 w
    \item Preliminary Literature Preview --- 150 w
    \item Methodology --- 550 w
    \item Conclusion --- 200 w
    \item References
\end{enumerate}
        --- 1600 w

\section*{RAW Abstract}
The Blockchain technology is commonly associated with Bitcoin, since it was the
earliest popular use of the Blockchain technology. As the technologies evolved,
the number of various blockchains with different kinds of applications had
drastically risen. The huge number of blockchains can be explained by the usage
of the variety of cryptographic algorithms and protocols in them.\\
This study examined all known (up to day) common cryptographic algorithms and
protocols, which are being used nowadays.\\
The algorithms used in blockchains were classified by time complexity, space
complexity, and the resistance to hacking.\\
There is no ``the most secure'' algorithm, nor the ``the fastest'' algorithm.
Scrypt hashing is faster than SHA-256 hashing, but less secure to cracking
(although there was no evidence of any case of cracking this algorithm yet).
There are DLs, which are not blockchains and use completely different
technology, like DAG in IOTA cryptovalue. These kinds of algorithms basically
have the same functionality, but different metrics must be used in order to
measure them. There are also algorithms, which are not yet known to public.\\

\section{Introduction}
Recent years have witnessed a growing academic interest in cryptocurrencies,
blockchains and distributed ledgers$^{\autoref{sec:ta}}$. The past ten years
have shown that the advance in the field of studying distributed ledgers and
cryptocurrencies is enormous, compared to any other field in the past. One of
the results of the growing interest is thousands of researches carried out by
scientists and individuals. These researches initiated the creation of hundreds
cryptocurrencies and other uses of distributed ledgers. Some cryptocurrencies
repeat each other in the form of algorithms (hashing, crypto-) and protocols
used, but their spectrum is vast. Relatively little research has been carried
out on the algorithms used, and even less on protocols. The generalisability of
much published research on this issue is problematic. While most websites provide
non-scientific explanations, there is one study [1]TODOcitation, that does provide
some aggregation of information on algorithms and protocols in distributed
ledgers. However, firstly, a number of technical and essential questions remain
about the classification of algorithms and protocols for distributed ledgers,
and secondly, it had only been carried in 2014, which makes the information in
it obsolete. New algorithms, modern applications of old algorithms, new
cryptocurrencies and their protocols are released every year, and nowadays
there is quite a room for expancing in that area. This prospective study is
designed to investigate the use of modern algorithms and protocols in
distributed ledgers, and to make their up-to-date classification. The goal is
to extend the existent classification with new algorithms and protocols,
focusing on the broad scope of technologies, and not on their number.
Ideally, our detailed classification should reflect actual condition in the
field of distributed ledgers. The last, but not least aim of the project, is to
collect all implemented algorithms described, or code them if they have not
been programmed yet, in a complete library for free usage. This library will
serve as a ``toolbox'' for a programmer, who wants to build his/her own
cryptoledger, combining known algorithms and protocols from the developed
library.
% -----------------------------------------
% 333 words
% -----------------------------------------

% Part of the aim of this project is to develop software that is compatible with
% The past thirty years have seen increasingly rapid advances in the field of ...

\section{Technical Aspects}\label{sec:ta}
In order to proceed further, a definition of a distributed ledger, as well as
the difference between distributed ledgers and blockchains must be given. A
blockchain differs from a traditional spreadsheet or another ledger in that it
is a decentralized, distributed ledger, stored in a distributed databases of
network devices. People refer to it as ``distributed'' because no single entity
manages a distributed ledger system on its own. The
ledger is distributed across a network of computers, also known as ``nodes'', and
each involved party has access to the ledger. This access allows all parties to
receive real-time status updates on transactions which occur within the network
of nodes.

Each record of a transaction in a blockchain is represented by a timestamped
``block''. Whenever a new block is generated on a blockchain, the system
appends this block to the previous block using this blockchain’s unique
algorithm. The visual result of the process is one of a ``chain'' of blocks.
Hence the term ``blockchain''.

Not all distributed ledgers are blockchains. There are types of distributed
ledgers, which represent DAGs (Directed Acyclic Graphs), or some other
company-related data structure, rather than the chain of blocks. But, the
majority of distributed ledgers serve to the purpose of maintaining the
viability of cryptocurrencies. It does not mean that every Distributed Ledgers
Technology is used for creating cryptocurrencies. Some of them are used for
different purposes, which are not covered in this paper.

% -----------------------------------------
% 239 words
% -----------------------------------------

\section{Problem Statement}
% Спорное (ноблинкрасивое) утверждение. А ДОКАЖИ ЧТО РИЛ БЫЛA recognized?!
The importance of creating a comprehensive classification of cryptographic
algorithms and protocols for distributed ledgers has recently been recognized
by the industry. However, the existing classification is expired, and there are
no any comprehensive study on them in the online sources. It may cause
difficult problems to overcome for a user, who wants to work with distributed
ledgers (e.g.\ cryptocurrencies). A {target user}\label{user} in terms of this research is a
user, who is concerned about security and cryptography, who's desire is to
reveal and understand algorithms, being used in technology he/she is going to
encounter with. If, when buying a cryptocurrency, an actor
wants to know what cryptographic algorithms and protocols it uses, he/she may
encounter difficulties finding that out and comparing it to other
cryptocurrencies' algorithms and protocols. These are the consequences of
modern tendency for websites to be consumer-oriented and provide only crisp and
eye-appealing information. A major problem with their approach is that their
primary goal is to ``sell'' it's content, and only after it goes the
% можно ль так эмоционально "takes ages"?
information delivery. Consequently, it takes ages for average user to find out
underlying algorithms in a particular distributed ledger.

Limitations of modern online ``blockchain creators'' is another problem for our
target user. Platforms, which deliver (e.g.\ ``cryptocurrency creation''
service), on average provide up to 3 algorithms and even less protocols. In
fact, in certain cases they do not provide user the ability to choose any
protocol. Taking that into account, our target user will not be able to choose
algorithms and protocol for his own ledger from the whole variety. The
limitations that those services put the user in are problematic.

In summary, there is a need for a better understanding of today's
cryptoledger's infrastructure and a structured approach in identifying and
collecting the majority of cryptographic algorithms and protocols that are
being used widely today. More specifically, the following research questions
need to be addressed:
\begin{itemize}
    \item What are the typical cryptographic algorithms found in various distributed ledgers;
    \item What are the typical cryptographic protocols found in various distributed ledgers;
    \item How to classify these algorithms for easier understanding and representing;
    \item What are the current industry practices as well as research advancements in each algorithm and protocol addressed;
    \item How to unify the knowledge gathered about cryptographic algorithms and protocols into one neat place;
    \item How to gather implemented algorithms and protocols into a programming
          library, which will be easy to use for creating own cryptoledger
          purpose?
\end{itemize}

\section{Objectives}
An initial objective of the project was to identify, analyze and classify the
broad scope of cryptographic algorithms and protocols in distributed ledgers.
The main aspect of the objective is the broad scope of algorithms and protocols
itself, not the quantity of distributed ledgers analyzed. It means that (e.g.\
some cryptocurrencies) may not be reviewd if algorithms and protocols they use
have already been classified. This goal also includes the implementation of the
programming library with algorithms and protocols for creating own
cryptoledger.
Particularly, the study has the following sub-objectives:
\begin{itemize}
    \item To provide a comprehensive review of cryptographic algorithms found
          in various distributed ledgers;
    \item To provide a comprehensive review of cryptographic protocols found in
          various distributed ledgers;
    \item To develop a robust classification method for easier understanding
          and representing;
    \item To review current industry practices and researches in regards to
          cryptographic algorithms and protocols in distributed ledgers;
    \item To outline a unification method for working with all gathered
          information on the topic;
    \item To outline a programming method, language and methodology for
          providing the library with algorithms and protocols reviewed.
\end{itemize}
The result of this study will be valuable to the industry practitioners (our
target users\(^{\ref{user}}\)) as well as researchers. The data gathered will
be concise, comprehensive, structured and easy to read. The program library is
supposed to be a complete toolbox for creating own cryptoledger, choosing from
a variety of gathered implemented algorithms and protocols.


\section{Preliminary Literature Preview}
Previous research [1] TODOcitation has established ``current ctyptoprotocol
infrastructure''. By 2014, it was a complete and comprehensive diagram, showing
all algorithms and protocols used in most popular distributed ledgers up to
that time. The analysis is also well-structured, but generalisable. What is
missing from the past study is a comprehensive and structured approach
in determining algorithms and protocols for cryptoledgers analyzed. However,
this study is the closest study to the topic discussed, and can be a great
point for taking-off with this research. The whitepapers for such technologies
as [bitcoin], [blockchain], [iota], [plasma], [ripple], [tkcoin] will surely be
used thruought conducting the research. The value of these monographies
consists of the following facts:
\begin{itemize}
    \item Whitepapers usually reveal algorithms and protocols used in a
          particular distributed ledger;
    \item Whitepapers usually point out the source code of algorithms and
          protocols, or even entire ledger.
\end{itemize}

\section{Methodology}
The primary research method for this study is literature review and benchmark
analysis. Identification and classification of cryptographic algorithms and
protocols in distributed ledgers is the first and the main step toward this
paper's goal. This study will first review various types of distributed
ledgers, such as cryptocurrencies, and investigate their characteristics. Then,
based on understanding of the underlying structure of every ledger analyzed, a
list of all known algorithms and protocols will be produced. In the third stage
of this study, the new classification will be developed. Finally, the
programming part will take place, where the implementations of gathered
cryptographic algorithms and protocols will be assembled into a one complete
library. This library will be a toolbox for users who want to build their own
cryptoledger. This study will be conducted between November 2018 and March 2019.

\end{document}
